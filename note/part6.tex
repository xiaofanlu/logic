\section{Part 8}
\subsection{Predicate Formulas: Syntax}

The alphabet of predicate logic consists of:
\begin{itemize}
 \item symbols called \textit{object constants}, 
 \item symbols called \textit{predicate constants}, with a positive integer, called the \textit{arity}, assigned to each of them. 
 \item \textit{object variable} $x, y, z, x_1, y_1, z_1, x_2, y_2, z_2, \ldots$, \item the propositional connectives,
 \item the \textit{universal quantifier} $\forall$ and the existential quantifier $\exists$,
 \item the parentheses and the comma.
\end{itemize}
In this part of the course, a \textit{string} is a finite string of symbols in the alphabet of predicate logic. 

A \textit{term} is an object constant or an object variable. A string is called an \textit{atomic formula} if it has the form 
\begin{equation*}
P(t_1, \ldots, t_n)
\end{equation*}
where $P$ is a predicate constant of arity $n$ and $t_1, \ldots, t_n$ are terms. 

We define when a string is a \textit{(predicate) formula} recursively, as follows:
\begin{itemize}
\item every atomic formula is a formula, 
\item $\top$ and $\bot$ are formulas, 
\item if $F$ is a formula then $\neg F$ is a formula,  
\item for any binary connective $\odot$, if $F$ are $G$ are formulas then $(F \odot G)$ is a formula, 
\item for any quantifier $K$ and any variable $v$, if $F$ is a formula then $KvF$ is a formula.  
\end{itemize}

When we write predicate formulas, we will use the abbreviations introduced in Part 1 of these lecture notes. A string of the form $\forall v_1 \cdots \forall v_n$ will be written as $\forall v_1 \cdots v_n$ and similarly for the existential quantifier. 

An occurrence of a variable $v$ in a formula $F$ is bound if it belongs to a part of $F$ that has the form $KvG$; otherwise it is free. We say that a variable $v$ is \textit{free (bound)} in $F$ if at least one one occurrence of $v$ in $F$ is free(bound). A formula without free variable is called a \textit{closed formula}, or a \textit{sentence}. 

In the following problems, represent the given conditions by formulas that may contain the object constant $a$ and the ternary predicate constants $Sum$ and $Prod$. Think of the object variables as ranging over all nonnegative integers, and interpret the object and predicate constants as follows:
\begin{itemize}
\item $a$ represents 0, 
\item $Sum(x, y, z)$ represents the condition $x + y = z$, 
\item $Prod(x, y, z)$ represents the condition $xy = z$, 
\end{itemize}

\begin{enumerate}
\item[\textbf{Problem 52}] 
\begin{enumerate}[(a)]
\item $x = y$   $\qquad \Rightarrow \qquad $ $Sum(x, a, y)$. 
\item $x = 0$   $\qquad \Rightarrow \qquad $ $Sum(x, a, a) \quad Sum(x, x, x)$. 
\item $x = 1$   $\qquad \Rightarrow \qquad $ $\forall y \:Prod(x, y, y)$. \\
                       $\neg Sum(x, x, x) \wedge \forall z(\neg Sum(z, z, z) \rightarrow \exists y \: Sum(x, y, z))$
\end{enumerate}

\item[\textbf{Problem 53}] 
\begin{enumerate}[(a)]
\item $x \leq y$ $\qquad \Rightarrow \qquad $ $ \exists z \: Sum(x, z, y)$. 
\item $x < y$    $\qquad \Rightarrow \qquad $ $ \exists z \: (\neg Sum(z, z, z) \wedge Sum(x, z, y))$.
\item $x + 1 < y$   $\quad \Rightarrow \qquad $ $ \exists z \: (\neg Prod(z, z, z) \wedge Sum(x, z, y)$. 
\end{enumerate}

\item[\textbf{Problem 54}] 
\begin{enumerate}[(a)]
\item $x$ is even number; \\
%$ \exists y \forall z Prod (y, z, z) \wedge \exists x_0 Sum(y, y, x_0) \wedge \exists z \: Prod(z, x_0, x)$.
$\exists y \: Sum(y, y, x)$
\item the sum of any two odd numbers is even; \\
%$\forall x, y, x_0, y_0 \: \neg Prod(x_0, 2, x)  \neg Prod(y_0, 2, y)$.
$\forall x_1 x_2 \; (\neg \exists y \; Sum(y, y, x_1) \wedge \neg \exists y \; Sum(y, y, x_2)) \rightarrow \exists z \; (Sum (x_1, x_2, z) \wedge \exists y \; Sum(y, y, z))$ \\
$\forall x_1 x_2 \; (\neg \exists y \; Sum(y, y, x_1) \wedge \neg \exists y \; Sum(y, y, x_2)) \rightarrow \forall z \; (Sum (x_1, x_2, z) \rightarrow \exists y \; Sum(y, y, z))$


\item addition is commutative. \\
$\forall xy \; \exists z \; (Sum(x, y, z) \wedge Sum(y, x, z))$
\end{enumerate}

\item[\textbf{Problem 55}] Number $x$ can be represented as the sum of two complete squares. \\
$\exists y_1 y_2 \; (\exists z \; Prod(z, z, y_1) \wedge \exists z \; Prod(z, z, y_2) \wedge Sum(y_1, y_2, x))$

\item[\textbf{Problem 56}] Number $x$ is prime. \\
%$\forall y, z \: \neg Prod(y, z, x)$\\
$\neg \exists y (\neg (\forall z Prod(y,z,z)) \wedge \neg Sum(y,a,x) \wedge \exists z (Prod(y, z , x))) \wedge \neg Prod(x, x, x)$
\end{enumerate}

\newpage
\subsection{Predicate Formulas: Semantics}
The semantics of propositional formulas described in Part 1 of these lecture notes defines which truth value $F^I$ is assigned to a propositional formula $F$ by an interpretation $I$. Our goal is to extend this definition to predicate logic.

First we need to adapt the definition of an interpretation to predicate formulas. In predicate logic, an \textit{interpretation} $I$ consists of
\begin{itemize}
\item a non-empty set $|I|$, called the \textit{universe} of $I$,  
\item for every object constant $c$, an element $c^I$ of $|I|$, 
\item for every predicate constant $P$, a function $P^I$ from $|I|^n$ to $\{f, t\}$, where $n$ is the arity of $P$. 
\end{itemize}

For instance, the sentence before Problem 52 can be viewed as the definition of the interpretation $I$ such that
\begin{equation}
\label{inter1}
\begin{gathered}
|I| = \mathbf{N}, \\
a^I = 0, \\
Sum^I(\xi, \eta, \zeta) = 
\left\{
        \begin{array}{ccc}      
        &\mathrm{t}, \quad &\text{if }  \xi + \eta = \zeta, \\
        &\mathrm{f}, \quad &\text{otherwise}
        \end{array}\right. \\
Prod^I(\xi, \eta, \zeta) = 
\left\{
        \begin{array}{ccc}      
        &\mathrm{t}, \quad &\text{if }  \xi \eta = \zeta, \\
        &\mathrm{f}, \quad &\text{otherwise}
        \end{array}\right.       
\end{gathered}
\end{equation}
$(\xi, \eta, \zeta \in \mathbf{N})$. 

The result of the \textit{substitution} of a term $t$ for a variable $v$ in a formula $F$ is the formula $F_t^v$ obtained from $F$ by replacing each free occurrence of $v$ by $t$. 

Let $I$ be an interpretation. For any element $\xi$ of its universe $|I|$, select a new object constant $\xi^*$, called the \textit{name} of $\xi$. The interpretation $I$ can be extended to the new object constants by defining 
\begin{equation*}
(\xi^*)^I = \xi
\end{equation*}
for all $\xi \in |I|$. 

We define the truth value $F^I$ that is \textit{assgined} to $F$ by $I$ for a sentence $F$ that may contain names recursively, as follows:

\begin{itemize}
\item $P(t_1, \ldots, t_n)^I = P^I(t_1^I, \ldots, t_n^I)$,
\item $\top^I = \mathrm{t}$, $\bot^I = \mathrm{f}$, 
\item $(\neg F)^I = \neg (F^I)$ , 
\item $(F \odot G)^I = \odot(F^I, G^I)$ for every binary connective $\odot$,
\item $(\forall vF)^I = \mathrm{t}$ iff for all $\xi \in |I|, (F^v_{\xi^*})^I = \mathrm{t}$,
\item $(\exists vF)^I = \mathrm{t}$ iff for some $\xi \in |I|, (F^v_{\xi^*})^I = \mathrm{t}$.
\end{itemize}

As in propositional logic, we say that $I$ \textit{satisfies} $F$, and write $I \models F$ if $F^I = \mathrm{t}$. 


\begin{enumerate}
\item[\textbf{Problem 57}] Determine which of the sentences
\begin{enumerate}[(i)]
    \item $\exists x Sum(x, x, x)$,
    \item $\exists x \neg Sum(x, x, x)$, 
    \item $\forall x (Sum(x,x,x) \rightarrow \forall y Sum(x, y, y))$
\end{enumerate}
are satisfied by interpretation \eqref{inter1} . 


\begin{enumerate}[(i)]
    \item $\exists x Sum(x, x, x)$,  \\
       For $0 \in |I|, (Sum(x, x, x)^{x}_{0^*})^I = (Sum(0^*, 0^*, 0^*))^I$ \\
               $~$\hspace{51.4mm}                      $= Sum^I({0^*}^I,{0^*}^I,{0^*}^I) = Sum^I(0,0,0) = \mathrm{t}$ 
    \\
    \item $\exists x \neg Sum(x, x, x)$, \\
    For $1 \in |I|, (\neg Sum(x, x, x)^x_{1^*})^I = (\neg Sum(1^*, 1^*, 1^*))^I$ \\
             $~$\hspace{55.4mm}$ =\neg Sum^I({1^*}^I,{1^*}^I,{1^*}^I) = \neg Sum^I(1,1,1) = \mathrm{t}$ 
    \\
    \item $\forall x (Sum(x,x,x) \rightarrow \forall y Sum(x, y, y))$\\
    $(Sum(n^*, n^*, n^*))^I = Sum^I({n^*}^I,{n^*}^I,{n^*}^I) = Sum^I(n, n, n) = 
    \left\{
        \begin{array}{ccc}      
        &\mathrm{t}, \quad &\text{if }  n = 0, \\
        &\mathrm{f}, \quad &\text{otherwise}
        \end{array}\right.    $
    We only care about $x = 0$, \\
    $(Sum(0^*, m^*, m^*))^I = Sum^I({0^*}^I,{m^*}^I,{m^*}^I) = Sum^I(0, m, m) = \mathrm{t}$  \\
    Thus, $[\forall y Sum(0^*, y, y)]^I = t$, for all $m = |I|$. 
\end{enumerate}
\end{enumerate}


\newpage

\subsection{Logical Validity and Entailment}

In predicate logic, we say about a sentence that it is \textit{logically valid} if it is satisfied by all interpretations. 

\begin{enumerate}
\item[\textbf{Problem 58}] Determine whether the sentences
\begin{eqnarray}
    \label{58_1}
   \exists x (P(x) \wedge Q(x)) \rightarrow (\exists x P(x) \wedge \exists x Q(x)), \\
    \label{58_2}
   (\exists x P(x) \wedge \exists x Q(x)) \rightarrow \exists x (P(x) \wedge Q(x))
\end{eqnarray}
are logically valid. 
\begin{itemize}
\item \eqref{58_1} is logically valid. \\
For any interpretation $I$, we only need care about the case when $( \exists x (P(x) \wedge Q(x)))^I = \mathrm{t}$, otherwise, the whole sentence is $\mathrm{t}$. We have, for some $\xi \in |I|$, 
\begin{eqnarray*}
(((P(x) \wedge Q(x)))^x_{\xi^*})^I = & (P(\xi^*) \wedge Q(\xi^*))^I \\
                                   = & \wedge (P^I(\xi^{*I}), Q^I(\xi^{*I})) \\
                                   = & \wedge (P^I(\xi), Q^I(\xi))
                                   = & \mathrm{t}
\end{eqnarray*}
Thus, $P^I(\xi) = Q^I(\xi) = \mathrm{t}$. This is to say: \\
$(P(x)^x_{\xi^*})^I = P^I(\xi^{*I}) = P^I(\xi) = \mathrm{t}$ $\qquad$
$(Q(x)^x_{\xi^*})^I = Q^I(\xi^{*I}) = Q^I(\xi) = \mathrm{t}$ \\
which means $(\exists x P(x))^I = (\exists x Q(x))^I = \mathrm{t}$. Thus, 
\begin{equation*}
(\exists x P(x) \wedge \exists x Q(x))^I = \mathrm{t}
\end{equation*}
This is to say, \eqref{58_1} is satisfied by all interpretations. Thus, it is logically valid. \\
We can also prove it by natural deduction. 
\setcounter{c}{0}
\begin{table}[H]
\begin{center}
\begin{tabular}{llll}
A1. & $\exists x (P(x) \wedge Q(x)) $                                  & $\qquad$& \\
\xfl{A1 \Rightarrow \exists x (P(x) \wedge Q(x)) }                    {axiom}
A2. & $P(x) \wedge Q(x) $                                              & $\qquad$& \\
\xfl{A2 \Rightarrow P(x) \wedge Q(x) }                                {axiom}
\xfl{A2 \Rightarrow P(x) }                                            {$(\wedge E), \xfp{c}$}
\xfl{A2 \Rightarrow \exists x P(x) }                                  {$(\exists I), \xfp{c}$}
\xfl{A2 \Rightarrow Q(x) }                                            {$(\wedge E), 2$}
\xfl{A2 \Rightarrow \exists x Q(x) }                                  {$(\exists I), \xfp{c}$}
\xfl{A2 \Rightarrow \exists x P(x) \wedge \exists x Q(x) }            {$(\wedge I), 4, \xfp{c}$}
\xfl{A1 \Rightarrow \exists x P(x) \wedge \exists x Q(x) }            {$(\exists E), 1, \xfp{c}$}
\xfl{\Rightarrow \exists x (P(x) \wedge Q(x)) \rightarrow (\exists x P(x) \wedge \exists x Q(x))}     
                                                                      {$(\rightarrow I), \xfp{c}$}
\end{tabular}
\end{center}
\end{table}

\newpage
\item \eqref{58_2} is not logically valid. \\
We can find a counter example. Define interpretation $I$ as:
\begin{equation*}
\begin{gathered}
|I| = \{0, 1\}, \\
P^I(\xi) = 
\left\{
        \begin{array}{ccc}      
        &\mathrm{t}, \quad &\text{if }  \xi = 0, \\
        &\mathrm{f}, \quad &\text{otherwise}
        \end{array}\right. \\
Q^I(\xi) = 
\left\{
        \begin{array}{ccc}      
        &\mathrm{t}, \quad &\text{if }  \xi = 1, \\
        &\mathrm{f}, \quad &\text{otherwise}
        \end{array}\right.       
\end{gathered}
\end{equation*}
Clearly, 
\begin{itemize}
\item \hspace{2cm} for $0 \in |I|, (P(x)_{0*}^x)^I = P^I(0^{*I}) = P^I(0) = \mathrm{t}$
\item \hspace{2cm} for $1 \in |I|, (Q(x)_{1*}^x)^I = Q^I(1^{*I}) = Q^I(1) = \mathrm{t}$
\end{itemize}
Thus, $(\exists x P(x) \wedge \exists x Q(x))^I = \mathrm{t}$. However, $\exists x (P(x) \wedge Q(x))$ can not be satisfied by $I$. 
\begin{itemize}
\item for $0 \in |I|, ((P(x) \wedge Q(x))_{0*}^x)^I = \wedge (P^I(0^{*I}), Q^I(0^{*I}))$ \\
           $~$\hspace{53.2mm} $= \wedge (P^I(0), Q^I(0)) = \wedge (\mathrm{t}, \mathrm{f}) = \mathrm{f}$ 
\item for $1 \in |I|, ((P(x) \wedge Q(x))_{1*}^x)^I= \wedge (P^I(1^{*I}), Q^I(1^{*I}))$ \\
           $~$\hspace{53.2mm} $= \wedge (P^I(1), Q^I(1)) = \wedge (\mathrm{f}, \mathrm{t}) = \mathrm{f}$ 
\end{itemize}
This is to say, the above interpretation fails to satisfy \eqref{58_2}. In other word, \eqref{58_2} is not logical valid. 
\end{itemize}
\end{enumerate}

\noindent
The \textit{univeral closure} of a formula $F$ is the sentence $\forall v_1, \cdots, v_n F$, where $v_1, \cdots, v_n$ are all free variables of $F$. About a formula with free variables we say that it is \textit{logically valid} if its universal closure is logically valid. 

\begin{enumerate}
\item[\textbf{Problem 59}] For each of the formulas 
\begin{eqnarray}
     \label{59_1}
     P(x) \rightarrow \exists x P(x) \\ 
     \label{59_2}
     P(x) \rightarrow \forall x P(x) 
\end{eqnarray}
determine whether it is logically valid. 
\end{enumerate}
\begin{itemize}
\item \eqref{59_1} is logically valid. \\
\item \eqref{59_2} is not logically valid. \\
\end{itemize}

\vspace{6cm}

\noindent
A formula $F$ is \textit{equivalent} to a formula $G$ if the formula $F \leftrightarrow G$ is logically valid. 

\begin{enumerate}
\item[\textbf{Problem 60}] Determine whether the formula 
\begin{equation*}
      \forall x \exists y P(x, y)
\end{equation*}
is equivalent to 
\begin{equation*}
      \exists y \forall x P(x, y)
\end{equation*}
\textit{We need to determine whether $\forall x \exists y P(x, y) \leftrightarrow  \exists y \forall x P(x, y)$ is logically valid or not.} 
\begin{itemize}
\item $\forall x \exists y P(x, y) \rightarrow  \exists y \forall x P(x, y)$ is not logically valid.
We can find a counter example. Define interpretation $I$ as:
\begin{equation*}
\begin{gathered}
|I| = \{0, 1\}, \\
P^I(\xi, \eta) = 
\left\{
        \begin{array}{ccc}      
        &\mathrm{t}, \quad &\text{if }  \xi = \eta, \\
        &\mathrm{f}, \quad &\text{otherwise}
        \end{array}\right.     
\end{gathered}
\end{equation*} 
There are only two elements $0$ and $1$ in the universe $|I|$.
\begin{itemize}
\item for $0 \in |I|, \exists y (P(x, y)_{0*}^x)^I = (\exists y P(0^*, y))^I$, we can find $0 \in |I|$ that \\
$(P(0^*, y)_{0*}^y)^I = P^I(0^{*I}, 0^{*I}) = P^I(0, 0) = \mathrm{t}$,
\item for $1 \in |I|, \exists y (P(x, y)_{1*}^x)^I = (\exists y P(1^*, y))^I$, we can find $1 \in |I|$ that \\
$(P(1^*, y)_{1*}^y)^I = P^I(1^{*I}, 1^{*I}) = P^I(1, 1) = \mathrm{t}$. 
\end{itemize}
However, the above interpretation can not satisfy $\exists y \forall x P(x, y)$. 
\begin{itemize}
\item for $0 \in |I|, \forall x (P(x, y)_{0*}^y)^I = (\forall x P(x, 0^*))^I$, we can find $1 \in |I|$
that  \\
$(P(x, 0^*)_{1*}^x)^I = P^I(1^{*I}, 0^{*I}) = P^I(1, 0) = \mathrm{f}$,
\item for $1 \in |I|, \forall x (P(x, y)_{1*}^y)^I = (\forall x P(x, 1^*))^I$, we can find $0 \in |I|$  that \\
$(P(x, 1^*)_{0*}^x)^I = P^I(0^{*I}, 1^{*I}) = P^I(0, 1) = \mathrm{f}$. 
\end{itemize}

\item $\exists y \forall x P(x, y) \rightarrow  \forall x \exists y P(x, y)$ is logically valid.
We can prove by natural deduction.
\setcounter{c}{0}
\begin{table}[H]
\begin{center}
\begin{tabular}{llll}
A1. & $\exists y \forall x P(x, y)$                                   & $\qquad$& \\
\xfl{A1 \Rightarrow \exists y \forall x P(x, y) }                     {axiom}
A2. & $\forall x P(x, y)$                                              & $\qquad$& \\
\xfl{A2 \Rightarrow \forall x P(x, y) }                                {axiom}
\xfl{A2 \Rightarrow P(x, y)}                                          {$(\forall E), \xfp{c}$}
\xfl{A2 \Rightarrow \exists y P(x, y)}                                {$(\exists I), \xfp{c}$}
\xfl{A2 \Rightarrow \forall x \exists y P(x, y)}                      {$(\forall I), \xfp{c}$}
\xfl{A1 \Rightarrow \forall x \exists y P(x, y)}                      {$(\exists E), 1, \xfp{c}$}
\xfl{\Rightarrow \exists y \forall x P(x, y) \rightarrow  \forall x \exists y P(x, y)}     
                                                                      {$(\rightarrow I), \xfp{c}$}
\end{tabular}
\end{center}
\end{table}                                                                          
\end{itemize}
\end{enumerate}


\newpage

We say that a set $\Gamma$ of sentences \textit{entails} a sentence $F$, or that $F$ is \textit{logical consequence} of $\Gamma$, if every interpretation that satisfies all sentences in $\Gamma$ satisfies $F$. 

\begin{enumerate}
\item[\textbf{Problem 61}]  Determine whether 
\begin{equation}
      \label{61_1}
      \exists x P(x, x) 
\end{equation}
is a logical consequence of the sentence 
\begin{eqnarray}
       \label{61_2}
       &\forall x \exists y P(x, y)\\
       \label{61_3}
       & \forall xyz((P(x, y) \wedge P(y,z)) \rightarrow P(x, z)). 
\end{eqnarray}

$~$\\
We can find a counter example. Define interpretation $I$ as:
\begin{equation*}
\begin{gathered}
|I| = \mathbb{N}, \\
P^I(\xi, \eta) = 
\left\{
        \begin{array}{ccc}      
        &\mathrm{t}, \quad &\text{if }  \xi < \eta, \\
        &\mathrm{f}, \quad &\text{otherwise}
        \end{array}\right.     
\end{gathered}
\end{equation*} 
$I$ satisfies \eqref{61_2} because for all $\xi = n \in |I|$, we can always have $\eta = n + 1$, where \begin{center}
$((P(x, y)_{\xi^*}^x)_{\eta^*}^y)^I  = P^I(\xi^{*I}, \eta^{*I}) = P^I(n, n + 1) = \mathrm{t}$. 
\end{center}

We can show that $I$ also satisfies \eqref{61_3} as follow: \\
We only care about the case where 
$(P(\xi^*, \eta^*) \wedge P(\eta^*, \zeta^*))^I = \mathrm{t}$. \\
From definition, we have:
\begin{eqnarray*}
(P(\xi^*, \eta^*) \wedge P(\eta^*, \zeta^*))^I  
                           = & \wedge (P^I(\xi^{*I}, \eta^{*I}),  P^I(\eta^{*I}, \zeta^{*I}))\\
                           =  & \wedge (P^I(\xi, \eta),  P^I(\eta, \zeta)) = \mathrm{t}
\end{eqnarray*}
We have
\begin{equation*}
\left.
        \begin{array}{ccc}      
        P^I(\xi, \eta) = \mathrm{t} &  \Rightarrow  &  \xi < \eta  \\
        P^I(\eta, \zeta) = \mathrm{t} &  \Rightarrow  &  \eta < \zeta 
        \end{array}\right\} \Rightarrow
            \xi < \eta < \zeta  
\end{equation*}
Clearly, 
\begin{equation*}
(P(\xi^*, \zeta^*))^I = P^I(\xi^{*I}, \zeta^{*I}) = P^I(\xi, \zeta) = \mathrm{t}
\end{equation*}

However, $I$ doesn't satisfy \eqref{61_1}. 
\begin{equation*}
(P(\xi^*, \xi^*))^I = P^I(\xi^{*I}, \xi^{*I}) = P^I(\xi, \xi) = \mathrm{f}
\end{equation*}


\end{enumerate}
\newpage

\subsection{Introduction and Elimination Rules for Quantifiers}
We say that a term $t$ is \textit{substitutable} for a variable $v$ in a formula $F$ if 
\begin{itemize}
\item $t$ is a constant, or 
\item $t$ is a variable $w$, and no part of $F$ of the form $KwG$ contains an occurrence of $v$ which is free in $F$. 
\end{itemize}
\noindent
Here are the additional inference rules of predicate logic:

\begin{center}
\begin{minipage}[t]{0.3\textwidth}
\begin{equation*}
(\forall I) \quad \frac{ \Gamma \Rightarrow F}{\Gamma \Rightarrow \forall v F }
\end{equation*} \\
where $v$ is not a free variable of any formula in $\Gamma$. \\
\begin{equation*}
(\exists I) \quad \frac{ \Gamma \Rightarrow F_t^v}{\Gamma \Rightarrow \exists v F}
\end{equation*}\\
where $t$ is substitutable for $v$ in $F$
\end{minipage}
\begin{minipage}[t]{0.1\textwidth}
$\quad$
\end{minipage}
\begin{minipage}[t]{0.3\textwidth}
\begin{equation*}
(\forall E) \quad \frac{ \Gamma \Rightarrow \forall vF }{\Gamma \Rightarrow F_t^v}
\end{equation*} \\
where $t$ is substitutable for $v$ in $F$ \\
\begin{equation*}
(\exists E) \quad \frac{ \Gamma \Rightarrow \exists vF \quad \Delta, F \Rightarrow \Sigma }
{\Gamma , \Delta \Rightarrow \Sigma}
\end{equation*} \\
where $v$ is not a free variable of any formula in $\Delta, \Sigma$
\end{minipage}
\end{center}

\noindent
Prove the given formulas in the natural deduction system. \\\\
\begin{enumerate}
\setcounter{c}{0}
\item[\textbf{Problem 62}] $(P(a) \wedge \forall x (P(x) \rightarrow Q(x))) \rightarrow Q(a)$. 
\begin{table}[H]
\begin{center}
\begin{tabular}{llll}
A1. & $P(a) \wedge \forall x (P(x) \rightarrow Q(x))$        & $\qquad$ & \\
\xfl{A1 \Rightarrow P(a) \wedge \forall x (P(x) \rightarrow Q(x))}          {axiom}
\xfl{A1 \Rightarrow p(a)}                                                   {$(\wedge E), \xfp{c}$}
\xfl{A1 \Rightarrow \forall x (P(x) \rightarrow Q(x))}                      {$(\wedge E), \xfp{c}$}
\xfl{A1 \Rightarrow P(a) \rightarrow Q(a)}                                  {$(\forall E), \xfp{c}$}
\xfl{A1 \Rightarrow Q(a)}                                               {$(\rightarrow E), 2, \xfp{c}$}
\xfl{\Rightarrow (P(a) \wedge \forall x (P(x) \rightarrow Q(x))) \rightarrow Q(a)}                                               {$(\rightarrow I), \xfp{c}$}

\end{tabular}
\end{center}
\end{table}

\newpage
\setcounter{c}{0}
\item[\textbf{Problem 63}] $P(a) \rightarrow \neg \forall x \neg P(x)$. 
\begin{table}[H]
\begin{center}
\begin{tabular}{llll}
A1. & $P(a) $                                            & $\qquad$ & \\
\xfl{A1 \Rightarrow P(a)}                                {axiom}
A2. & $\forall x \neg P(x) $                             & $\qquad$ & \\
\xfl{A2 \Rightarrow \forall x \neg P(x)}                 {axiom}
\xfl{A2 \Rightarrow \neg P(a)}                           {$(\forall E), \xfp{c}$}
\xfl{A1,A2 \Rightarrow \bot}                             {$(\neg E), 1, \xfp{c}$}
\xfl{A1 \Rightarrow \neg \forall x \neg P(x)}            {$(\neg I), \xfp{c}$}
\xfl{\Rightarrow P(a) \rightarrow \neg \forall x \neg P(x)}     {$(\rightarrow I), \xfp{c}$}
\end{tabular}
\end{center}
\end{table}

\setcounter{c}{0}
\item[\textbf{Problem 64}] $\forall xy P(x,y) \rightarrow \forall x P(x, x)$. 
\begin{table}[H]
\begin{center}
\begin{tabular}{llll}
A1. & $\forall xy P(x,y) $                                  & $\qquad$& \\
\xfl{A1 \Rightarrow \forall x \forall y P(x,y) }            {axiom}
\xfl{A1 \Rightarrow \forall y P(x,y) }                      {$(\forall E), \xfp{c}$}
\xfl{A1 \Rightarrow  P(x, x) }                              {$(\forall E), \xfp{c}$}
\xfl{A1 \Rightarrow \forall x  P(x,x) }                     {$(\forall I), \xfp{c}$}
\xfl{\Rightarrow \forall xy P(x,y) \rightarrow \forall x P(x, x)}     {$(\rightarrow I), \xfp{c}$}
\end{tabular}
\end{center}
\end{table}


\setcounter{c}{0}
\item[\textbf{Problem 65}] $\forall x P(x) \leftrightarrow \forall y P(y)$. 
\begin{table}[H]
\begin{center}
\begin{tabular}{llll}
A1. & $\forall x P(x) $                               & $\qquad$ & \\
\xfl{A1 \Rightarrow \forall x P(x) }                  {axiom}
\xfl{A1 \Rightarrow P(y) }                            {$(\forall E), \xfp{c}$}
\xfl{A1 \Rightarrow \forall y P(y) }                  {$(\forall I), \xfp{c}$}
\xfl{\Rightarrow \forall x P(x) \rightarrow  \forall y P(y) }{$(\rightarrow I), \xfp{c}$}
\\
A2. & $\forall y P(y) $                               & $\qquad$ & \\
\xfl{A1 \Rightarrow \forall y P(y) }                  {axiom}
\xfl{A1 \Rightarrow P(x) }                            {$(\forall E), \xfp{c}$}
\xfl{A1 \Rightarrow \forall x P(x) }                  {$(\forall I), \xfp{c}$}
\xfl{\Rightarrow \forall y P(y) \rightarrow  \forall x P(x) }{$(\rightarrow I), \xfp{c}$}
\\
\xfl{\Rightarrow \forall x P(x) \leftrightarrow  \forall y P(y) }{$(\wedge I), 4, \xfp{c}$}
\end{tabular}
\end{center}
\end{table}

\newpage
\setcounter{c}{0}
\item[\textbf{Problem 66}] $\forall x P(x) \wedge \forall x Q(x) \leftrightarrow \forall x (P(x) \wedge Q(x))$. 
\begin{table}[H]
\begin{center}
\begin{tabular}{llll}
A1. & $\forall x P(x) \wedge \forall x Q(x) $          & $\qquad$ & \\
\xfl{A1 \Rightarrow \forall x P(x) \wedge \forall x Q(x)}       {axiom}
\xfl{A1 \Rightarrow \forall x P(x) }                            {$(\wedge E), \xfp{c}$}
\xfl{A1 \Rightarrow P(x) }                                      {$(\forall E), \xfp{c}$}
\xfl{A1 \Rightarrow \forall x Q(x) }                            {$(\wedge E), 1$}
\xfl{A1 \Rightarrow Q(x) }                                      {$(\forall E), \xfp{c}$}
\xfl{A1 \Rightarrow P(x) \wedge Q(x) }                          {$(\wedge I), 3, \xfp{c}$}
\xfl{A1 \Rightarrow \forall x (P(x) \wedge Q(x)) }              {$(\forall I), \xfp{c}$}
\xfl{\Rightarrow \forall x P(x) \wedge \forall x Q(x) \rightarrow \forall x (P(x) \wedge Q(x))}
                                                                {$(\rightarrow I), \xfp{c}$}

\\
A2. & $\forall x (P(x) \wedge Q(x)) $                           & $\qquad$ & \\
\xfl{A2 \Rightarrow \forall x (P(x) \wedge Q(x))}               {$(\forall I), \xfp{c}$}
\xfl{A2 \Rightarrow P(x) \wedge Q(x))}                          {$(\forall E), \xfp{c}$}
\xfl{A2 \Rightarrow P(x) }                                      {$(\wedge E), \xfp{c}$}
\xfl{A2 \Rightarrow \forall x P(x) }                            {$(\forall I), \xfp{c}$}
\xfl{A2 \Rightarrow Q(x) }                                      {$(\wedge E), 9$}
\xfl{A2 \Rightarrow \forall x Q(x) }                            {$(\forall I), \xfp{c}$}
\xfl{A2 \Rightarrow \forall x P(x) \wedge \forall x Q(x) }      {$(\wedge I), 11, \xfp{c}$}
\xfl{\Rightarrow \forall x (P(x) \wedge Q(x)) \rightarrow \forall x P(x) \wedge \forall x Q(x)}
                                                                {$(\rightarrow I), \xfp{c}$}

\\
\xfl{\Rightarrow \forall x P(x) \wedge \forall x Q(x) \leftrightarrow \forall x (P(x) \wedge Q(x)) }
                                                                {$(\wedge I), 4, \xfp{c}$}
\end{tabular}
\end{center}
\end{table}

\vspace{-5mm}
\setcounter{c}{0}
\item[\textbf{Problem 66\nicefrac{1}{2}}] $\forall x P(x) \vee \forall x Q(x) \leftrightarrow \forall x (P(x) \vee Q(x))$. 
\begin{table}[H]
\begin{center}
\begin{tabular}{llll}
A1. & $\forall x P(x) \vee \forall x Q(x) $                      & $\qquad$ & \\
\xfl{A1 \Rightarrow \forall x P(x) \vee \forall x Q(x)}          {axiom}
A2. & $\forall x P(x)$                                           & $\qquad$ & \\
\xfl{A2 \Rightarrow \forall x P(x)}                              {axiom}
\xfl{A2 \Rightarrow P(x) }                                      {$(\forall E), \xfp{c}$}
\xfl{A2\Rightarrow P(x) \vee Q(x) }                             {$(\vee I), \xfp{c}$}
A3. & $\forall x Q(x)$                                           & $\qquad$ & \\
\xfl{A3 \Rightarrow \forall x Q(x)}                              {axiom}
\xfl{A3 \Rightarrow Q(x) }                                      {$(\forall E), \xfp{c}$}
\xfl{A3\Rightarrow P(x) \vee Q(x) }                             {$(\vee I), \xfp{c}$}
\xfl{A1 \Rightarrow P(x) \vee Q(x) }                            {$(\vee E), 1, 4, \xfp{c}$}
\xfl{A1 \Rightarrow \forall x (P(x) \vee Q(x))}                            {$(\forall I), \xfp{c}$}
\xfl{\Rightarrow \forall x P(x) \vee \forall x Q(x) \rightarrow \forall x (P(x) \vee Q(x))}
                                                                {$(\rightarrow I), \xfp{c}$}
\end{tabular}
\end{center}
\end{table}

Counter Example for $\leftarrow$: \\
$|I| = {a, b}, P^I(a) = \mathrm{t}, P^I(b) = \mathrm{f}, Q^I(a) = \mathrm{f}, Q^I(b) = \mathrm{t}$ 
\begin{eqnarray*}
((P(x) \vee Q(x))^x_{\eta^*})^I = & (P(\eta^*) \vee Q(\eta^*))^I \\
                                = & \vee(P^I (\eta^{*I}), Q^I (\eta^{*I})) \\
                                = & \vee(P^I (\eta), Q^I (\eta)) 
                                = &    \left\{
                                        \begin{array}{ccc}      
                                        &\mathrm{t}, \quad &\text{if }  \eta = a, \\
                                        &\mathrm{t}, \quad &\text{if }  \eta = b.
                                        \end{array}\right.    
\end{eqnarray*}
However, 
\begin{eqnarray*}
(P(x)^x_{\eta^*})^I = & (P(\eta^*))^I \\
                    = & P^I (\eta^{*I}) \\
                    = & P^I (\eta) 
                    = & \left\{
                    \begin{array}{ccc}      
                    &\mathrm{t}, \quad &\text{if }  \eta = a, \\
                    &\mathrm{f}, \quad &\text{if }  \eta = b.
                    \end{array}\right.                        
\end{eqnarray*}
Thus, $\forall x (P(x)^x_{\eta^*})^I = \mathrm{f}$. Similarly, we can get $\forall x (Q(x)^x_{\eta^*})^I = \mathrm{f}$. Thus, 
\begin{equation*}
(\forall x P(x) \vee \forall x Q(x))^I = \mathrm{f}
\end{equation*}
This means that
\begin{equation*}
( \forall x (P(x) \vee Q(x)) \rightarrow (\forall x P(x) \vee \forall x Q(x)))^I = \mathrm{f}
\end{equation*}
Thus rule doesn't hold for the above interpretation. 

\newpage

\setcounter{c}{0}
\item[\textbf{Problem 67}] $\forall x P(x) \vee Q(a) \leftrightarrow \forall x (P(x) \vee Q(a))$. 
\begin{table}[H]
\begin{center}
\begin{tabular}{llll}
A1. & $\forall x P(x) \vee Q(a)$                                   & $\qquad$ & \\
\xfl{A1 \Rightarrow \forall x P(x) \vee Q(a)}                      {axiom}
A2. & $\forall x P(x) $                                            & $\qquad$ & \\
\xfl{A2 \Rightarrow \forall x P(x)}                                {axiom}
\xfl{A2 \Rightarrow P(x)}                                          {$(\forall E), \xfp{c}$}
\xfl{A2 \Rightarrow P(x) \vee Q(a)}                                {$(\vee I), \xfp{c}$}
\xfl{A2 \Rightarrow \forall x (P(x) \vee Q(a))}                    {$(\forall I), \xfp{c}$}
A3. & $Q(a)$                                                       & $\qquad$ & \\
\xfl{A3 \Rightarrow Q(a)}                                         {axiom}
\xfl{A3 \Rightarrow P(x) \vee Q(a)}                                {$(\vee I), \xfp{c}$}
\xfl{A3 \Rightarrow \forall x (P(x) \vee Q(a))}                    {$(\forall I), \xfp{c}$}
\xfl{A1 \Rightarrow \forall x (P(x) \vee Q(a))}                    {$(\vee E), 1, 5, \xfp{c}$}
\xfl{\Rightarrow \forall x P(x) \vee Q(a) \rightarrow \forall x (P(x) \vee Q(a))}      {$(\rightarrow I), \xfp{c}$}
\\
A4. & $\forall x (P(x) \vee Q(a))$                                   & $\qquad$ & \\
\xfl{A4 \Rightarrow \forall x (P(x) \vee Q(a))}                      {axiom}
\xfl{A4 \Rightarrow P(x) \vee Q(a)}                                  {$(\forall E), \xfp{c}$}
\xfl{\Rightarrow Q(a) \vee \neg Q(a)}                                {axiom}
A5. & $Q(a)$                                                         & $\qquad$ & \\
\xfl{A5 \Rightarrow Q(a)}                                            {axiom}
\xfl{A5 \Rightarrow \forall x P(x) \vee Q(a)}                        {axiom}
A6. & $ \neg Q(a)$                                                   & $\qquad$ & \\
\xfl{A6 \Rightarrow \neg Q(a)}                                       {axiom}
\xfl{A5, A6 \Rightarrow \bot}                                        {$(\neg E), \xfp{c}$}
\xfl{A5, A6 \Rightarrow P(x)}                                        {$(C), \xfp{c}$}
\xfl{P(x) \Rightarrow P(x)}                                          {axiom}
\xfl{A4, A6 \Rightarrow P(x)}                                        {$(\vee E), 12, 18, \xfp{c}$}
\xfl{A4, A6 \Rightarrow \forall x P(x)}                              {$(\forall I), \xfp{c}$}
\xfl{A4, A6 \Rightarrow \forall x P(x) \vee Q(a)}                    {$(\vee I), \xfp{c}$}
\xfl{A4 \Rightarrow \forall x P(x) \vee Q(a)}                        {$(\vee E), 13, 15, \xfp{c}$}
\\
\xfl{\Rightarrow \forall x (P(x) \vee Q(a)) \rightarrow \forall x P(x) \vee Q(a) }      {$(\rightarrow I), \xfp{c}$}


\end{tabular}
\end{center}
\end{table}

\newpage

\setcounter{c}{0}
\item[\textbf{Problem 68}] $(P(a) \vee P(b)) \rightarrow \exists x P(x)$. 
\begin{table}[H]
\begin{center}
\begin{tabular}{llll}
A1. & $P(a) \vee P(b) $                                            & $\qquad$ & \\
\xfl{A1 \Rightarrow P(a) \vee P(b)}                                {axiom}
A2. & $P(a) $                                                      & $\qquad$ & \\
\xfl{A2 \Rightarrow P(a)}                                          {axiom}
\xfl{A2 \Rightarrow \exists x P(x)}                                {$(\exists I), \xfp{c}$}
A3. & $P(b) $                                                      & $\qquad$ & \\
\xfl{A3 \Rightarrow P(b)}                                          {axiom}
\xfl{A3 \Rightarrow \exists x P(x)}                                {$(\exists I), \xfp{c}$}
\xfl{A1 \Rightarrow \exists x P(x)}                                {$(\vee E), 1, 3, \xfp{c}$}
\xfl{\Rightarrow (P(a) \vee P(b)) \rightarrow \exists x P(x)}      {$(\rightarrow I), \xfp{c}$}
\end{tabular}
\end{center}
\end{table}

\setcounter{c}{0}
\item[\textbf{Problem 69}] $(\exists x P(x) \wedge \forall x (P(x) \rightarrow Q(x))) \rightarrow \exists x Q(x)$. 
\begin{table}[H]
\begin{center}
\begin{tabular}{llll}
A1. & $(\exists x P(x) \wedge \forall x (P(x) \rightarrow Q(x)))$  & $\qquad$ & \\
\xfl{A1 \Rightarrow (\exists x P(x) \wedge \forall x (P(x) \rightarrow Q(x)))}     {axiom}
\xfl{A1 \Rightarrow \exists x P(x)}                                           {$(\wedge E), \xfp{c}$}
\xfl{A1 \Rightarrow \forall x (P(x) \rightarrow Q(x))}                        {$(\wedge E), \xfp{c}$}
\xfl{A1 \Rightarrow P(x) \rightarrow Q(x)}                                    {$(\forall E), \xfp{c}$}
\xfl{ P(x) \Rightarrow  P(x)}                                                 {axiom}
\xfl{A1, P(x) \Rightarrow  Q(x)}                                              {$(\rightarrow E), \xfp{c}, 4$}
\xfl{A1, P(x) \Rightarrow  \exists Q(x)}                                      {$(\exists I), \xfp{c}$}
\xfl{A1 \Rightarrow \exists x Q(x)}                                           {$(\exists E), 2, \xfp{c}$}
\xfl{\Rightarrow (\exists x P(x) \wedge \forall x (P(x) \rightarrow Q(x)))}   {$(\rightarrow I), \xfp{c}$}
\end{tabular}
\end{center}
\end{table}

\newpage
\setcounter{c}{0}
\item[\textbf{Problem 70}] $\exists x P(x) \leftrightarrow \exists y P(y)$. 
\begin{table}[H]
\begin{center}
\begin{tabular}{llll}
A1. & $\exists x P(x) $                                       & $\qquad$ & \\
\xfl{A1 \Rightarrow \exists x P(x) }                          {axiom}
A2. & $P(x)$                                                  & $\qquad$ & \\
\xfl{A2 \Rightarrow P(x) }                                    {axiom}
\xfl{A2 \Rightarrow \exists y P(y) }                          {$(\exists I), \xfp{c}$}
\xfl{A1 \Rightarrow \exists y P(y) }                          {$(\exists E), 1, \xfp{c}$}
\xfl{\Rightarrow \exists x P(x) \rightarrow  \exists y P(y) } {$(\rightarrow I), \xfp{c}$}
\\
A3. & $\exists y P(y) $                                       & $\qquad$ & \\
\xfl{A3 \Rightarrow \exists y P(y) }                          {axiom}
A2. & $P(y)$                                                  & $\qquad$ & \\
\xfl{A2 \Rightarrow P(y) }                                    {axiom}
\xfl{A2 \Rightarrow \exists x P(x) }                          {$(\exists I), \xfp{c}$}
\xfl{A1 \Rightarrow \exists x P(x) }                          {$(\exists E), 1, \xfp{c}$}
\xfl{\Rightarrow \exists y P(y) \rightarrow  \exists x P(x) } {$(\rightarrow I), \xfp{c}$}
\\
\xfl{\Rightarrow \exists x P(x) \leftrightarrow  \exists y P(y) }{$(\wedge I), 4, \xfp{c}$}
\end{tabular}
\end{center}
\end{table}

\newpage
\setcounter{c}{0}
\item[\textbf{Problem 71}] $\neg \exists x P(x) \leftrightarrow \forall x \neg P(x)$. 
\begin{table}[H]
\begin{center}
\begin{tabular}{llll}
A1. & $\neg \exists x P(x) $                                  & $\qquad$ & \\
\xfl{A1 \Rightarrow \neg \exists x P(x) }                     {axiom}
A2. & $P(x)$                                                  & $\qquad$ & \\
\xfl{A2 \Rightarrow P(x) }                                    {axiom}
\xfl{A2 \Rightarrow \exists x P(x) }                          {$(\exists I), \xfp{c}$}
\xfl{A1, A2 \Rightarrow \bot }                                {$(\neg E), 1, \xfp{c}$}
\xfl{A1 \Rightarrow \neg P(x) }                               {$(\neg I), \xfp{c}$}
\xfl{A1 \Rightarrow \forall x \neg P(x) }                     {$(\forall I), \xfp{c}$}
\xfl{\Rightarrow \neg \exists x P(x) \rightarrow \forall x \neg P(x)} {$(\rightarrow I), \xfp{c}$}
\\
A3. & $\forall x \neg P(x) $                                  & $\qquad$ & \\
\xfl{A3 \Rightarrow \forall x \neg P(x)}                      {axiom}
\xfl{A3 \Rightarrow \neg P(x)}                                {$(\forall E), \xfp{c}$}
\xfl{P(x) \Rightarrow P(x) }                                  {axiom}
\xfl{A3, P(x) \Rightarrow \bot}                               {$(\neg E), \xfp{c}$, 9}
A4. & $\exists x P(x)$                                        & $\qquad$ & \\
\xfl{A4 \Rightarrow \exists x P(x)}                           {axiom}
\xfl{A3, A4 \Rightarrow \bot}                                 {$(\exists E), \xfp{c}, 11$}
\xfl{A3 \Rightarrow \neg \exists x P(x)}                      {$(\neg I), \xfp{c}$}
\xfl{\Rightarrow  \forall x \neg P(x) \rightarrow  \neg \exists x P(x)} {$(\rightarrow I), \xfp{c}$}
\\
\xfl{\Rightarrow \neg \exists x P(x) \leftrightarrow \forall x \neg P(x) }{$(\wedge I), 4, \xfp{c}$}
\end{tabular}
\end{center}
\end{table}

\item[\textbf{Problem 71\nicefrac{1}{2}}] $\neg \forall x P(x) \leftrightarrow \exists x \neg P(x)$. 

\end{enumerate}






















 




 
 