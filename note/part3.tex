\section{Part 3}
\subsection{Positive Programs}
A \textit{positive rule} is a propositional formula of the form $F \rightarrow G$, where $F$ and $G$ contain no connectives other than $\top, \bot, \wedge, \text{and } \vee$. The antecedent $F$ is called the \textit{body} of the rule, and the consequent $G$ is called its \textit{head}. Rules are often written with the head on the left and body on the right: $G \leftarrow F$. A rule of the form $G \leftarrow \top$ is often identified with its head $G$. Rules of the form $\bot \leftarrow F$ are called \textit{constraints} and are usually written as $\leftarrow F$. 

A \textit{positive (logic) program} is a set of positive rules. 

A positive rule is \textit{flat} if (i) its head is $\bot$, or an atom, or a disjunction of several atoms, and (ii) its body is $\top$, or an atom, or a conjunction of several atoms. Any propositional formula can be transformed into an equivalent set of flat positive rules by converting it to conjunctive normal form and then rewriting each of its simple disjunctions
\begin{equation*}
A_1 \vee \cdots A_m \vee \neg A_{m + 1} \vee \cdots \vee \neg A_n
\end{equation*}
as the rule
\begin{equation*}
A_1 \vee \cdots A_m \vee \leftarrow A_{m + 1} \wedge \cdots \wedge A_n
\end{equation*}
*The expression in the head is understood as $\bot$ if $m = 0$; the expression in the body is understood as $\top$ if $m = n$. 

\subsection{Minimal Models}
In the theory of logic programs it is customary to identify an interpretation with the set of atoms to which it assigns the value $t$. For instance, the interpretation that assigns $t$ to the atom $p$ and $f$ to all other atoms can be viewed as the singleton $\{p\}$. 

A \textit{model} of a set $\Gamma$ of formulas is an interpretation that satisfies all formulas in $\Gamma$. A model $M$ of $\Gamma$ is \textit{minimal} if no proper subset of $M$ is a model of $\Gamma$. For example, the program
\begin{equation}
\begin{gathered}
p, \\
q \leftarrow r
\end{gathered}
\end{equation}
has 3 models:
\begin{equation*}
 \{p\}, \{p, q\}, \{p, q, r\};
\end{equation*}
only the first of them is minimal. 

\begin{enumerate}
\item[\textbf{Problem 21}] Find all models of the program
\begin{gather*}
p\vee q, \\
q \vee r, \\
q \leftarrow p \wedge r. 
\end{gather*}
Determine which of these models are minimal. \\
\begin{gather*}
 \{p, q, r\} \\
 \{p,q\}, \{q, r\} \\
  \{q\}
\end{gather*}




\item[\textbf{Problem 22}] Consider the positive program consisting of the rules $A \vee B$ for all pairs of distinct atoms $A, B$. If the total number of atoms is $n$, then how many models does this program have? How many of them are minimal? \\
Claim: A model of the above positive program must have at least $n - 1$ atoms. 
\begin{proof}
Suppose we have a model $M$ with less than $n - 1$ atoms. $\exists$ two distinct atoms $A$ and $B$, where $A \vee B$ is one of the rules while $A \not \in M, B \not \in M$. Thus, $(A \vee B)^M = \vee (A^M, B^M) = f \vee f = f$. Thus, $M$ failed to satisfy all rules. Contradiction. 
\end{proof}
From the above claim, we can see that the minimal model should have $n - 1$ atoms. $C_{n}^{n - 1} = n$. There is one more model which contains all the atoms. Thus, the total number of models is $n + 1$. 


\item[\textbf{Problem 23}] Let $\Gamma$ be a positive program such that the head of each of its rules is an atom. Show that the intersection of all models of $\Gamma$ is the only minimal model of $\Gamma$. \\
Claim: Given a positive program $\Gamma$ such that the head of each of its rules is an atom, for two model $M_1, M_2$, $M_1 \cap M_2$ is model as well.  
\begin{proof} $~$
\begin{itemize}
\item For the head $p$ of each rule in the form $p \leftarrow F$ :
\begin{itemize}
\item If $p \in M_1 \cap M_2$, then the rule is satisfied by $M_1 \cap M_2$. 
\item Otherwise, suppose $p \not \in M_i$, then $F^{M_i} = f$. As $F$ contains no connectives other than $\top, \bot, \wedge$ and $\vee$, a subset of $M_i$ doesn't satisfy $F$ as well (Contrapositive of Problem 24). Thus, $F^{M_1 \cap M_2} = f$, the rule is also satisfied by $M_1 \cap M_2$. 
\end{itemize}
\item For the head $p$ of each rule in the form $p \leftarrow \top$:  \\
$p \in M_1, p \in M_2$, thus $p \in M_1 \cap M_2$, the rule is satisfied under $M_1 \cap M_2$
\end{itemize}
\end{proof}
By induction, $\cap M_i$ is model, which is also the subset of all models. Thus, the intersection of all models of $\Gamma$ is the only minimal model. 


\end{enumerate}



\noindent If $\Gamma_1, \Gamma_2$ are sets of formulas such that $\Gamma_1 \subseteq \Gamma_2$ then every model of $\Gamma_2$ is a model of $\Gamma_1$. But we cannot assert, in general, that every minimal model of $\Gamma_2$ is a minimal model of $\Gamma_1$. For instance, if we add the rule $r \leftarrow p$ to program (1) then it will get a new minimal model, $\{p, q, r\}$. In this sense, the concept of a minimal model is "nonmonotonic."

\begin{enumerate}
\item[\textbf{Problem 24}] Let $F$ be a formula containing no connectives other than $\top, \bot, \wedge, \vee$, and let $M_1, M_2$ be sets of atoms such that $M_1 \subseteq M_2$. Show that if $M_1$ satisfies $F$ then so does $M_2$. 

\begin{proof} Use structural induction: 
\begin{itemize}
\item If $F$ is an atom, $M_1$ satisfies $F$ then so does $M_2$;
\item $\top, \bot$ is trivial;
\item There is no $\neg$ in F, we don't need to consider this case;
\item Suppose given $M_1$ satisfy both $F$ and $G$, and $M_1 \subseteq M_2$, we have $M_2$ satisfy both $F$ and $G$ as well. 
    \begin{itemize}
        \item For $F \wedge G$, if $(F \wedge G)^{M_1} = \wedge (F^{M_1}, G^{M_1}) = t$, then $F^{M_1} = t$ and $G^{M_1} = t$. From induction hypothesis, we have $F^{M_2} = t$ and $G^{M_2} = t$. Thus, $(F \wedge G)^{M_2} = \wedge (F^{M_2}, G^{M_2}) = t$, induction hypothesis holds true. 
        \item For $F \vee G$, $(F \vee G)^{M_1} = \vee (F^{M_1}, G^{M_1}) = t$, then $F^{M_1} = t$ or $G^{M_1} = t$. From induction hypothesis, we have $F^{M_2} = t$ or $G^{M_2} = t$. Thus, $(F \vee G)^{M_2} = \vee (F^{M_2}, G^{M_2}) = t$,induction hypothesis holds true;
        \item there is no $\rightarrow$ in F, we don't need to consider this case. 
    \end{itemize}
\end{itemize}
\end{proof}

\item[\textbf{Problem 25}] For any positive program $\Gamma$ and any constraint $\leftarrow F$, an interpretation $M$ is a \textbf{minimal} model of $\Gamma \cup \{\leftarrow F \}$ iff $M$ is a \textbf{minimal} model of $\Gamma$ and does not satisfy $F$. 
\begin{itemize}
    \item[$\leftarrow$] If $M$ is a \textbf{minimal} model of $\Gamma$ and does not satisfy $F$, prove that $M$ is a \textbf{minimal} model of $\Gamma \cup \{\leftarrow F \}$.
     \begin{itemize}
         \item $M$ is a model of $\Gamma \cup \{\leftarrow F \}$: \\
         Clearly, $M$ satisfies $\Gamma$. $M$ doesn't satisfy $F$, thus $M$ satisfy $\{\leftarrow F \}$. Thus,  $M$ satisfies $\Gamma \cup \{\leftarrow F \}$.
         \item $M$ is a minimal model of $\Gamma \cup \{\leftarrow F \}$:\\
          Suppose there is a subset $M'$ of $M$ which is a model for $\Gamma \cup \{\leftarrow F \}$ as well. Thus, $M'$ must also be a model of $\Gamma$. However, we already know that $M$ is a minimal model of $\Gamma$. Contradiction. Thus, $M$ is a minimal model of $\Gamma \cup \{\leftarrow F \}$. 
     \end{itemize}
     \item[$\rightarrow$] If $M$ is a \textbf{minimal} model of $\Gamma \cup \{\leftarrow F \}$, prove that $M$ is a \textbf{minimal} model of $\Gamma$ and does not satisfy $F$. 
     \begin{itemize}
         \item Sicne $\{\leftarrow F \}^M = t$, $F^M = f$, thus, $M$ does not satisfy $F$.
         \item Clearly, $M$ must be a model of $\Gamma$. 
         \item Suppose there is a subset $M'$ of $M$ which is a model for $\Gamma$. From the contrapositive of Problem 24, $M'$ doesn't satisfy $F$ as well. Thus, from $\leftarrow$, we know that $M'$ is a model of $\Gamma \cup \{\leftarrow F \}$. However, we already know that $M$ is a \textbf{minimal} model of $\Gamma \cup \{\leftarrow F \}$. Contradiction. Thus, $M$ is a \textbf{minimal} model of $\Gamma$. 
     \end{itemize}
\end{itemize}



\item[\textbf{Problem 26}] If $M$ is a minimal model of a positive program $\Gamma$ then every atom from $M$ occurs in the head of one of the rules of $\Gamma$. 

%Idea: Suppose $M$ is a minimal model and one of the atoms from $M$ does not occur in the head of any of the \textbf{flat} rules. By removing such atom from M, we will not change the interpretation of each head. However, the body can't be changed from false to true (The body can only be $\top$, or an atom, or a conjunction of several atoms). For all other cases, the new model satisfies $\Gamma$, thus, $M$ is not a minimal model. Contradiction. 

\begin{proof} Suppose there is an atom $p$ in minimal model $M$ that doesn't occur in the head of any of the rules. Define $M' = M \backslash \{p\}$. For any rule $H \leftarrow B$ in $\Gamma$, if 
    \begin{itemize}
        \item if $M$ satisfies $H$, $M'$ satisfies $H$ as well, so $M'$ satisfies $H \leftarrow B$. 
        \item if $M$ doesn't satisfy $H$, $M'$ doesn't neither. Since $M$ satisfies $H \leftarrow B$, $M$ must don't satisfy $B$. From Problem 24, $M'$ doesn't satisfy $B$ as well.  Thus, $M'$ satisfy rule $H \leftarrow B$. Since we already know $M$ is a minimal model. Contradiction. 
    \end{itemize}
\end{proof}
\end{enumerate}

\subsection{Quiz 4}
\noindent
Let $\Gamma$ be a positive program containing the rule $p \leftarrow q$. Show that if $p$ doesn't occur in the heads of the other rules of $\Gamma$ then every minimal model of $\Gamma$ satisfies the formula $p \leftrightarrow q$. 

\begin{proof}
For any minimal model $M$ of $\Gamma$, 
\begin{itemize}
   \item If $q \in M$, since the rule $p \leftarrow q$ must be satisfied by $M$, then $p \in M$ as well. Clearly, M satisfies $p \leftrightarrow q$.  
   \item If $q \not \in M$, the rule $p \leftarrow q$ is satisfied. Suppose $p \in M$, define $M' = M \backslash \{p\}$. Clearly, the rule $p \leftarrow q$ is satisfied by $M'$ as well. For all other rules $H \leftarrow B$ in $\Gamma$:
   \begin{itemize}
       \item If $M$ satisfies $H$, since $p$ doesn't occur in $H$, $M'$ satisfies $H$ as well. Then $M'$ satisfies this rule. 
       \item If $M$ doesn't satisfy $H$, since $M$ satisfies $H \leftarrow B$, $M$ must not satisfy $B$. By problem 24, $M'$, a subset of $M$, doesn't satisfy $B$ as well. Thus, $M'$ also satisfies this rule. 
   \end{itemize}
   This is to say, $M'$ satisfies $\Gamma$, which contradicts that $M$ is a minimal model of $\Gamma$. Thus, our assumption that $p \in M$ must be false. This is to say, if $q \not \in M$, $p \not \in M$ as well. Clearly, M satisfies $p \leftrightarrow q$. 
\end{itemize}
%To sum up, if $p$ doesn't occur in the heads of the other rules of $\Gamma$ then every minimal model of $\Gamma$ satisfies the formula $p \leftrightarrow q$. 

\end{proof}


\subsection{Application: Graph Coloring}
We would like to assign one of three colors to each vertex of a graph $G$ so that adjacent vertices will have different colors. This problem can be reduced to generating a minimal model of a positive program. The atoms in this program are expressions of the form $color(v,c)$, where v is a vertex of $G$ and $c$ is one of the colors $blue, red, yellow$. The program consists of the rules
\begin{equation}
    color(v, blue) \vee color(v, red) \vee color(v, yellow)
    \label{rule2}
\end{equation}

for all vertices $v$, and the constrains
\begin{equation}
    \leftarrow color(v, c) \wedge color(v', c)
    \label{rule3}
\end{equation}
for all pairs $v, v'$ of adjacent vertices of $G$ and all colors $c$. 
\begin{enumerate}
\item[\textbf{Problem 27}] A set $X$ of atoms is a minimal model of program \eqref{rule2}, \eqref{rule3} iff
    \begin{itemize}
         \item for every vertex $v$ there is exactly one color $c$ such that $color(v, c) \in X$, and
         \item for every pair of atoms $color(v, c), color(v', c')$ from $X$, if $v$ is adjacent to $v'$, then $c \neq c'$.
    \end{itemize}

\begin{proof} $~$
$X$ of atoms is a minimal model of program (2), (3) \\
$\iff$  \{ by problem 25\}
	\begin{itemize}
		\item $X$ is a minimal model of (2) for all vertex $v$, and 
       \item $X$ does not satisfy $color(v,c) \wedge color(v',c)$ for all pairs $v$, $v'$ of adjacent vertices of $G$ and all colors $c$.       
	\end{itemize}
$\iff$  \{ by definition\}
	\begin{itemize}
		\item for every vertex $v$ there is exactly one color $c$ such that $color(v,c) \in X$, and 
        \item for every pair of atoms $color(v,c), color(v',c')$ from $X$, if $v$ is adjacent to $v'$ then $c \neq c'$.
    \end{itemize}
\end{proof}

\newpage


%
%
%\begin{itemize}
%	\item[$\rightarrow$]
%		$X$ of atoms is a minimal model of program (2), (3): \\
%		By Problem 25, $X$ is a minimal model of (2) for all vertices $v$, and $X$ does not satisfy $color(v,c) \wedge color(v',c)$ for all pairs $v$, $v'$ of adjacent vertices of $G$ and all colors $c$.\\
%		$\Rightarrow$
%	\begin{itemize}
%		\item for every vertex $v$ there is exactly one color $c$ such that $color(v,c) \in X$, and 
%        \item for every pair of atoms $color(v,c), color(v',c')$ from $X$, if $v$ is adjacent to $v'$ then $c \neq c'$.
%	\end{itemize}
%
%	\item[$\leftarrow$]
%	\begin{itemize}
%		\item for every vertex $v$ there is exactly one color $c$ such that $color(v,c) \in X$ $\Rightarrow$ $X$ is a minimal model of (2).
%        \item for every pair of atoms $color(v,c), color(v',c')$ from $X$, if $v$ is adjacent to $v'$ then $c \neq c'$. $\Rightarrow$ $X$ does not satisfy $color(v,c) \wedge color(v',c)$\\
%			By Problem 25, A set $X$ of atoms is a minimal model of program $(2) \cup (3)$. 
%	\end{itemize}    
%\end{itemize}    

\end{enumerate}