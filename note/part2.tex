\section{Part 2}
\subsection{Adequate Sets of Connectives}
\begin{enumerate}
\item[\textbf{Problem 11}] For any formula, there exists an equivalent formula that contains no connectives other than (i) $\wedge$ and $\neg$; (ii) $\vee$ and $\neg$.  (structural induction)
    \begin{enumerate} [(i)]
        \item $\wedge$ and $\neg$ 
            \begin{itemize}
                \item There is no connective in atoms. Thus every atom has property $P$. 
                \item $\top \sim p \vee \neg p \sim \neg (p \wedge \neg p)$; $\bot \sim p \wedge \neg p$. Thus $\top$ and $\bot$ has property $P$. 
                \item By induction hypothesis, there exist $F' \sim F$, that contains no connective other than $\wedge$ \text{and} $\neg$.  Clearly, $\neg F \sim \neg F'$.  So $\neg F$ has property P. 
                \item $F \wedge G \sim F' \wedge G'$; \\
                 $F \vee G \sim F' \vee G' \sim \neg (\neg F' \wedge \neg G')$; (\textit{De Morgan's Law}) \\
                 $ F \rightarrow G \sim F' \rightarrow G' \sim \neg (F' \wedge \neg G')$;\\
                  Thus, for any binary connective $\odot$, if formulas $F$ and $G$ has property $P$, then so does $(F \odot G)$. 
            \end{itemize}
       Then we can conclude that property $P$ holds for all formulas.   
       
          \item $\vee$ and $\neg$ 
            \begin{itemize}
                \item There is no connective in atoms. Thus every atom has property $P$. 
                \item $\top \sim p \vee \neg p$; $\bot \sim \neg (p \vee \neg p) $. Thus $\top$ and $\bot$ has property $P$. 
                \item By induction hypothesis, there exist $F' \sim F$, that contains no connective other than $\wedge$ \text{and} $\neg$.  Clearly, $\neg F \sim \neg F'$.  So $\neg F$ has property P. 
                \item $F \wedge G \sim F' \wedge G' \sim \neg (\neg F' \vee \neg G') $; (\textit{De Morgan's Law}) \\
                 $F \vee G \sim F' \vee G'$;  \\
                 $ F \rightarrow G \sim F' \rightarrow G' \sim \neg (F' \wedge \neg G') \sim \neg F' \vee G'$ \\
                 Thus, for any binary connective $\odot$, if formulas $F$ and $G$ has property $P$, then so does $(F \odot G)$. 
            \end{itemize}
       Then we can conclude that property $P$ holds for all formulas.    
    \end{enumerate}
    
\item[\textbf{Problem 12}] For any formula, there exists an equivalent formula that contains no connectives other than (i) $\rightarrow $ and $\neg$; (ii) $\rightarrow $ and $\bot$.  (structural induction)
    \begin{enumerate} [(i)]
        \item $\rightarrow $ and $\neg$ 
            \begin{itemize}
                \item There is no connective in atoms. Thus every atom has property $P$. 
                \item $\top \sim p \rightarrow p$; $\bot \sim \neg(p \rightarrow p)$. Thus $\top$ and $\bot$ has property $P$. 
                \item By induction hypothesis, there exist $F' \sim F$, that contains no connective other than $\rightarrow $ \text{and} $\neg$. Clearly, $\neg F \sim \neg F'$.  So $\neg F$ has property P. 
                \item $F \wedge G \sim F' \wedge G' \sim F' \wedge \neg (\neg G') \sim \neg ( F' \rightarrow \neg G') $; \\
                 $F \vee G \sim F' \vee G' \sim \neg (\neg F' \wedge \neg G') \sim \neg 
                 F' \rightarrow G' $ ;  \\
                 $F \rightarrow G \sim F' \rightarrow G'$;\\
                  Thus, for any binary connective $\odot$, if formulas $F$ and $G$ has property $P$, then so does $(F \odot G)$. 
            \end{itemize}
       Then we can conclude that property $P$ holds for all formulas.   
       
          \item $\rightarrow $ and $\bot$ 
            \begin{itemize}
                \item There is no connective in atoms. Thus every atom has property $P$. 
                \item $\top \sim p \rightarrow p \: (\bot \rightarrow p) $; $\bot$ is trivial. Thus $\top$ and $\bot$ has property $P$. 
                \item By induction hypothesis, there exist $F' \sim F$, that contains no connective other than $\rightarrow $ \text{and} $\bot$.  Clearly, $\neg F \sim \neg F' \sim F' \rightarrow \bot $.  So $\neg F$ has property P. 
                \item $F \wedge G \sim F' \wedge G' \sim \neg ( F' \rightarrow \neg G') \sim [F' \rightarrow (G' \rightarrow \bot)] \rightarrow \bot$; \\
                 $F \vee G \sim F' \vee G' \sim \neg F' \rightarrow G' \sim (F' \rightarrow \bot) \rightarrow G'$;  \\
                 $ F \rightarrow G \sim F' \rightarrow G' $\\
                 Thus, for any binary connective $\odot$, if formulas $F$ and $G$ has property $P$, then so does $(F \odot G)$. 
            \end{itemize}
       Then we can conclude that property $P$ holds for all formulas.    
    \end{enumerate}

\item[\textbf{Problem 13}] Any propositional formula equivalent to $\neg p$ contains $\neg$ or $\bot$.  \\
Lemma: Any propositional formula containing neither $\neg$ nor $\bot$ under the interpretation that all the atoms are evaluated true will be interpreted as true.
\begin{proof} Proof by structural induction. 
\begin{enumerate}[i]
	\item Atom. Trivially.
	\item $\top$. $\top^I = t$.
	\item  $F\odot G$. Suppose	$F^I = t$, $G^I = t$, 
		\begin{itemize}
			\item $(F \wedge G)^I = (F^I \wedge G^I) = t$; \\
			\item $(F \vee G)^I = (F^I \vee G^I) = t$; \\
			\item $(F \rightarrow G)^I = (F^I \rightarrow G^I) = t$. 
		\end{itemize}
\end{enumerate}
By the structural induction, the lemma follows.\\
However, any propositional formula equivalent to $ \neg p$ can not be interpreted as true under the interpretation that all the atoms are evaluated true.\\
Thus, Any propositional formula equivalent to $\neg p$ contains $\neg$ or $\bot$.
\end{proof}

\end{enumerate}

\subsection{Normal Forms} %Note 3, page 5}
A \textit{literal} is an atom or the negation of an atom. A propositional formula is said to be \textit{negation normal form} if 
\begin{itemize}
    \item it contains no connectives other than conjunction, disjunction, and negation, and 
    \item every negation in it is part of a literal. 
\end{itemize}

\begin{enumerate}
\item[\textbf{Problem 14}] Any formula is equivalent to a formula in negation normal form.  
\begin{proof}
Use structure induction. 
\begin{itemize}
\item $F$ is atom $a$, $\neg F = \neg a$, both $F$ and $\neg F$ are in NNF.  
\item if $F$ is $\bot$, $ F \sim p \wedge \neg p$, $\neg F \sim p \vee \neg p$. Similar for the case where $F = \top$. Both $F$ and $\neg F$ are in NNF. 
\item Let $F$ and $\neg F$ have NNF. The negation of them are still NNF. 
\item Suppose $F, \neg F, G, \neg G$ have NNF, this is to say, we have $F \sim F', \neg F \sim F'', G \sim G', \neg G \sim G''$, where $F', F'', G', G''$ are in NNF. 
    \begin{itemize}
        \item $F \wedge G \sim F' \wedge G'$,  \quad $\neg (F \wedge G) \sim \neg F \vee \neg G \sim F'' \vee G''$;
        \item $F \vee G \sim F' \vee G'$,  \quad $\neg (F \vee G) \sim \neg F \wedge \neg G \sim F'' \wedge G''$;
        \item $F \rightarrow G \sim \neg F \vee G \sim F'' \vee G'$ ; \quad $\neg F \rightarrow \neg G \sim F \vee \neg G \sim F' \vee G''$. 
        
    \end{itemize}
\end{itemize}
\end{proof}
\end{enumerate}

\noindent
A $simple~ conjunction$ is a formula of the form $L_1 \wedge \cdots \wedge L_n (n \geq 1)$, where $L_1, \ldots, L_n$ are literals. A formula is in \textit{disjunctive normal form (DNF)} if it has the form $C_1 \vee \cdots \vee C_m (m \geq 1)$, where $C_1, \ldots C_m$ are simple conjunctions. 

\begin{enumerate}
\item[\textbf{Problem 15}] Any formula is equivalent to a formula in disjunctive normal form. 
\begin{proof}
For any formula $F$, let all the interpretations that satisfy $F$ are $I_1, I_2, \ldots, I_n$, and define set $S = \{I_1, I_2, \ldots, I_n\}$. \\
By Problem 5, there $\exists$ a formula $F'$ such that for all interpretations $I$, $I \models F'$ iff $I \in S$, where $F' = F_1' \vee \ldots \vee F_n'$, where $F'_i$ is the formula that $I_i$ is the only interpretation satisfying it. From problem 4, we know that $F'_i$ is a simple conjunction of literals. 
\begin{itemize}
	\item if $I \in S$ (i.e. $I \models F$), then $I \models F'$;
	\item if $I \notin S$ (i.e. $I \nvDash F$), then $I \nvDash F'$.
\end{itemize}

For all interpretations $I$, $F^I = F'^{I}$, so $F$ is equivalent to $F'$. Since $F'$ is in the form of DNF, F is equivalent to a formula in DNF.
\end{proof}

\end{enumerate}

\noindent
A $simple~ disjunction$ is a formula of the form $L_1 \vee \cdots \vee L_n (n \geq 1)$, where $L_1, \ldots, L_n$ are literals. Simple disjunctions are also called \textit{clauses}. 
\noindent
A formula is in \textit{conjunctive normal form (CNF)} if it has the form $D_1 \wedge \cdots \wedge D_m (m \geq 1)$, where $D_1, \ldots D_m$ are simple disjunctions. 

\begin{enumerate}

\item[\textbf{Problem 16}] Let $F$ be a formula in disjunctive normal form. Show that $\neg F$ is equivalent to a formula in conjunctive normal form. 
\begin{proof}
Since $F$ is in DNF, $F = C_1 \vee \cdots \vee C_m ~(m \geq 1)$, where $C_i = L_1 \wedge \cdots \wedge L_n$. \\
We have $\neg F \sim \neg C_1 \wedge \cdots \wedge \neg C_m ~(m \geq 1)$, where $\neg C_i \sim \overline{L_1} \vee \cdots \vee \overline{L_n}$. \\
Define $D_i = \neg C_i$, thus, $\neg F$ has the form $D_1 \wedge \cdots \wedge D_m (m \geq 1)$, where $D_1, \ldots, D_m$ are simple disjunctions. \\
Thus, $\neg F$ is equivalent to a formula in conjunctive normal form. 
\end{proof}

\item[\textbf{Problem 17}] Any formula is equivalent to a formula in conjunctive normal form.   \\
$\forall F $ \\
$~\quad \sim \qquad$  \{$\forall I, F^I = \neg (\neg (F^I)) = (\neg \neg F)^I$ \}\\
    $~\qquad \neg \neg F$  \\
$~\qquad \sim \qquad$   \{by Problem 15, $\neg F \sim G $ in DNF\}  \\ 
    $~\qquad \quad \neg G$  \\
 $~\qquad \quad \sim$   \{by Problem 16, $\neg G \sim H$ in CNF\}  \\ 
    $~\qquad \qquad H$ 
\end{enumerate}

\subsection{Satisfiability and Entailment}
A \textit{set} $\Gamma$ is \textit{satisfiable} if there exists an interpretation that satisfies all formulas in $\Gamma$, and \textit{unsatisfiable} otherwise. 

\begin{enumerate}
\item[\textbf{Problem 18}] A set $\Gamma$ of literals is satisfiable iff there is no atom $A$ for which both $A$ and $\neg A$ belong to $\Gamma$. 
\begin{itemize}
\item[$\rightarrow$] If a set $\Gamma$ of literals is satisfiable, then there $\exists$ an interpretation $I$ that satisfies all formulas in $\Gamma$. Clearly, for any atom $A$, $A^I$ and $(\neg A)^I$ can not be satisfied at the same time. Thus, there is no atom $A$ for which both $A$ and $\neg A$ belong to $\Gamma$.
\item[$\leftarrow$] If there is no atom $A$ for which both $A$ and $\neg A$ belong to $\Gamma$. Thus, we can find interpretation $I$ for any atom $A$ based on the following rule:
\begin{gather*}
A^I =  \left\{
        \begin{array}{c}      
        t \qquad \text{if} \: A \in \Gamma \\
        f \qquad \text{if} \; \neg A \in \Gamma \\
        f \qquad \text{otherwise}
        \end{array}\right.
\end{gather*}
\end{itemize}
Thus for any literal $L \in \Gamma$, we have $I(L) = t$. Clearly, this set $\Gamma$ of literal is satisfiable. 

\end{enumerate}



\noindent A set $\Gamma$ of formulas \textit{entails} a formula $F$ (symbolically, $\Gamma \models F$), if every interpretation that satisfies all formulas in $\Gamma$ satisfies $F$ also. 

\begin{enumerate}
\item[\textbf{Problem 19}] For any formulas $F_1, \ldots, F_n, G$, the following conditions are equivalent:
\begin{enumerate}[(1)]
\item $F_1, \ldots, F_n \models G$, 
\item $(F_1 \wedge \cdots \wedge F_n) \rightarrow G$ is tautology, 
\item the set $\{F_1, \ldots, F_n, \neg G \}$ is unsatisfiable. 
\end{enumerate}
\begin{enumerate}[(1)]
\item By definition, $\forall I s.t. F_1^I = t, \ldots, F_n^I = t$, we have $G^I = t$. 
\item $(F_1 \wedge \cdots \wedge F_n) \rightarrow G$ is tautology means that $\forall I s.t. (F_1 \wedge \cdots \wedge F_n)^I = t$, we have $G^I = t$. From problem 3, we know that $(F_1 \wedge \cdots \wedge F_n)^I = t \sim F_1^I = t, \ldots, F_n^I = t$. This is (1). Thus we have $(1) \Longleftrightarrow (2)$. 
\item the set $\{F_1, \ldots, F_n, \neg G \}$ is unsatisfiable means that there $\not \exists I s.t. F_1^I = t, \ldots, F_n^I = t, \neg G^I = t$. This is to say, $\forall I s.t. F_1^I = t, \ldots, F_n^I = t$, we have $\neg G^I = f$. Thus, $G^I = t$. This turn out to be (1). Thus we have $(1) \Longleftrightarrow (3)$. 
\end{enumerate}
Thus, we have $ (1) \Longleftrightarrow (2) \Longleftrightarrow (3)$. 
\end{enumerate}

\subsection{Clausification}
To \textit{clausify} a formula $F$ means to find a formula 
$F'$ that may contain some new atoms, not occurring in $F$, such that
\begin{itemize}
\item $F'$ is in conjunctive normal form,
\item any interpretation satisfying $F'$ satisfies $F$, and 
\item any interpretation satisfying $F$ can be extended to the new atoms so that it will satisfy $F'$. 
\end{itemize}
Here is an algorithm for clausifying a propositional formula: \\
\begin{algorithm}[H]
 \Begin{
     $\Gamma \leftarrow \emptyset$ \\
     \While{F is not CNF}{
          $A \leftarrow$ a new atom \\
          $G \leftarrow$ a minimal non-literal subformula of F \\
          $F \leftarrow$ the result of replacing G in F by A \\
          $\Delta \leftarrow$ the set of clauses of the CNF of $A
               \leftrightarrow  G$ \\
          $\Gamma \leftarrow \Gamma \cup \Delta$
      }
      \Return the conjunction of F with clauses $\Gamma$
}
\end{algorithm}

\begin{enumerate}
\item[\textbf{Problem 20}] (i) Apply this algorithm to the formula $p \vee \neg (q \rightarrow r)$. (ii) Determine whether this formula is equivalent to the result of its clausification. \\
(ii). The formula is not equivalent to the result of its clausification for the latter has new atom whose interpretation is not defined in the interpretation for the original formula. 
\end{enumerate}


\subsection{Quiz 3}
\noindent
The set of \textit{canonical} propositional formulas is defined recursively:
\begin{itemize}
\item every literal is a canonical formula;
\item if formulas $F$ and $G$ are canonical then the formulas $F \vee G$ and $F \rightarrow G$ are canonical also. 
\end{itemize}
Every formula is equivalent to a canonical formula. True or false?\\

\noindent
False, $\bot$ is not equivalent to a canonical formula. 
\begin{proof}
Use structural induction over the definition of canonical formula. 
\begin{itemize}
\item every literal can not be equivalent to $\bot$;
\item suppose formulas $F$ and $G$ are not equivalent to $\bot$. There $\exists$ an interpretation $I$, that $G^I = t$. \\
Thus, 
\begin{eqnarray*}
(F \vee G)^I =& \vee(F^I, G^I) =& t \\
(F \rightarrow G)^I =& \rightarrow (F^I, G^I) =& t 
\end{eqnarray*}
Clearly, neither $(F \vee G)$ nor $(F \rightarrow G)$ is equivalent to $\bot$. 
\end{itemize}
Then we can conclude that no canonical formula is equivalent to $\bot$. 
\end{proof}

\newpage