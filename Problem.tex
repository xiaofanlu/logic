\documentclass[12pt]{article}
\pagestyle{empty}

\usepackage{graphicx}
\usepackage{float}
\usepackage{times}
\usepackage{color}
\usepackage{listings}
\usepackage{enumerate}
\usepackage{amsmath}
\usepackage{amsthm}
\usepackage{fancyhdr}
\usepackage{titlesec}
\usepackage{ascii}
\usepackage[T1]{fontenc}
\usepackage{datetime}
\usepackage[top=1.5in, bottom=1.5in, left=1in, right=1in]{geometry}

\usepackage{algorithm2e}
  
\providecommand{\e}[1]{\ensuremath{\times 10^{#1}}}

\pagestyle{fancy} 
\renewcommand{\headrulewidth}{0pt}
\shortdate
\fancyhead[LE, RO]{\textit{\date{\currenttime} \today}}
\fancyhead[LO]{\textit {Xiaofan Lu}}



\titleformat{\section}{\bf\Large\centering}{\thesection}{1em}{}
  
\begin{document}
\title{CS388L Homework Solution}
\author{Xiaofan Lu, xl4326}
\date{\vspace{-3ex}}
\maketitle

\section*{Propositional Formulas: Syntax  \colorbox{green}{$~$ }}
\begin{enumerate}
\item[\textbf{Problem 1}] A formula cannot contain two binary connectives next to each other. True or false?\\
Prove by \textit{structural induction}. Define property $P$ as above. 
\begin{itemize}
\item There is no connectives in atoms. Thus every atom has property $P$. 
\item There is no connectives in $\top$ and $\bot$. Thus $\top$ and $\bot$ has property $P$. 
\item $\neg$ is an unitary connective. Thus if a formula $F$ has property $P$. Then so does $\neg F$. 
\item A formula neither stars nor ends with a binary connective. Thus if formulas F and G don't contain two binary connective next to each other. So does $(F \odot G)$. 
\end{itemize}
\noindent Then we can conclude that property $P$ holds for all formulas. 

\item[\textbf{Problem 2}] If a formula contains more than one character then its last character is not an atom. True or false?     (False. Consider $\neg p$)  \\
How about the first character? \\
True. Prove by \textit{structural induction}. Define property $P$ as above. 
\begin{itemize}
\item An atom has only one character. 
\item $\top$ and $\bot$ has only one character as well. 
\item $\neg F$ does not start with an atom. 
\item $(F \odot G)$ does not start with an atom as well.   
\end{itemize}
\noindent Then we can conclude that property $P$ holds for all formulas. 
\end{enumerate}




\section*{Propositional Formulas: Semantics \colorbox{green}{$~$ }}
\begin{enumerate}
\item[\textbf{Problem 3}] For any formulas $F_1, \ldots, F_n~(n \geq 1)$ and any interpretation $I$, 
\begin{eqnarray*}
 (F_1 \wedge \cdots \wedge F_n)^I &= t~~ \text{iff} ~~ F_1^I = \cdots = F_n^I &= t  \\
 (F_1 \vee \cdots \vee F_n)^I &= f~~ \text{iff} ~~ F_1^I = \cdots = F_n^I &= f 
\end{eqnarray*}

\noindent
Define: $G_n = F_1 \wedge \cdots \wedge F_n$. \\
$\bullet$ When $n = 1$, $G_1^I = F_1^I$, thus $G_1^I = t \iff F_1^I = t$  \\
$\bullet$ Let $k \in N$ and suppose $G_k^I = t \iff F_1^I = \cdots = F_k^I = t$, then
\begin{equation*}
    G_{k+1}^I = \left(G_k \wedge F_{k+1} \right)^I = \wedge \left(G_k^I, F_{k+1}^I \right)
\end{equation*}
\vspace{-30pt}
\begin{eqnarray*}
    G_{k+1}^I = t &\iff &G_k^I = t \And F_{k+1}^I = t  \\
    G_{k+1}^I = t &\iff & F_1^I = \cdots = F_k^I = F_{k+1}^I = t
\end{eqnarray*}
\hspace{10pt}Thus, the induction hypothesis holds true for $n = k+1$. \\
$\bullet$ By the principle of induction, $G_n^I = t \iff F_1^I = \cdots = F_n^I = t$, which is \\
\begin{equation*}
(F_1 \wedge \cdots \wedge F_n)^I = t~~ \text{iff} ~~ F_1^I = \cdots = F_n^I = t
\end{equation*}

\item[\textbf{Problem 4}] For any interpretation $I$, there exists a formula $F$ such that $I$ is the only interpretation satisfying $F$. 

For atom $p_i : 1 \leq i \leq n$, define 
$F_i=\left\{
    \begin{array}{c l}      
    p_i & p_i^I = t\\
    \neg p_i & p_i^I = f
\end{array}\right.$
Thus, $F = F_1 \wedge \cdots \wedge F_n$. \\
From \textbf{problem 3}, we know that $I$ satisfies $F$. \\
For any other interpretation $I'$. There must $\exists k \in N$ that $p_k^{I} = \neg p_k^{I'}$. Thus, $F_k^{I'} = f$, clearly, $I'$ doesn't satisfy $F$. 


\item[\textbf{Problem 5}] For any set $S$ of interpretations there exists a formula $F$ such that for all interpretation $I$, $I \vdash F ~~\text{iff}~~ I \in S$. \\
Student Solution: \\
If set S is empty, $F = \bot$. Else, let $S = \{ I_i :1 \leq i \leq n \}$ be a set of interpretation. For each $I_i$, let $F_i$, s.t. $I_i$ is the only interpretation satisfying $F_i$ (by Problem 4).  \\
If define $F = (F_1 \vee \cdots \vee F_n)$, since $F_i^{I_i} = t, F^{I_i} = t$. Thus, $I \in S \Rightarrow I \vdash F$. \\
Now, suppose $I' \notin S$, by construction, $\forall i, F_i^{I'} = f$. Thus,  $I \notin S \Rightarrow I \not\vdash F$. \\
Thus, $I \in S \iff I \vdash F$.\\

Proffessor Solution: \\
$I \vdash F$ \\
$~\quad \iff$  \{Construction of F \}\\
    $~\qquad I \vdash F_1 \vee \cdots \vee F_n $  \\
$~\qquad \iff$   \{by Problem 3\}  \\ 
    $~\qquad \quad$for some $i$, $I \vdash F_i$  \\
 $~\qquad \quad \iff$   \{by Problem 4, choice of $F_i$\}  \\ 
    $~\qquad \qquad$for some $i$, $I = I_i$  \\
 $~\qquad \qquad \iff$   \{set notation \}  \\ 
 $~\qquad \qquad \quad I \in \{  I_i :1 \leq i \leq n\}$   \\ 

\end{enumerate}

\section*{Tautologies and Equivalence \colorbox{green}{$~$ }}
A propositional formula F is a \textit{tautology} if every interpretation satisfies $F$.

\begin{enumerate}
\item[\textbf{Problem 6}] Determine which of the formulas 
\begin{gather*}
(p \rightarrow q) \vee  (q \rightarrow p) \\
((p \rightarrow q) \rightarrow  p) \rightarrow p \\
((p \rightarrow q) \rightarrow r) \rightarrow  ((p \rightarrow q) \rightarrow (p \rightarrow r ))  
\end{gather*}
    
\begin{table}[H]
\begin{center}
    \begin{tabular}{c c c c}
    \hline
    p & q & 1 & 2 \\ \hline
    f & f & t   & t   \\
    f & t & t   & t   \\
    t & f & t   & t   \\
    t & t & t   & t   \\ \hline
    \end{tabular}
\end{center}
\end{table}
\end{enumerate}

\noindent A formula $F$ is equivalent to a formula $G$ (symbolically, $F \sim G$) if, for every interpretation $I$, $F^I = G^I$. 

\begin{enumerate}
\item[\textbf{Problem 7}] (a) We know that conjunction and disjunction are associative: 
\begin{gather*}
(F \wedge G) \wedge H \sim F \wedge (G \wedge H) \\
(F \vee G) \vee H \sim F \vee (G \vee H)
\end{gather*}
Determine whether equivalence has a similar property: 
\begin{gather*}
 (F \leftrightarrow G) \leftrightarrow H \sim F \leftrightarrow (G \leftrightarrow H)
\end{gather*}
    \begin{gather*}
 F \rightarrow (G \wedge H) \sim (F \wedge G) \rightarrow (F \wedge H)
\end{gather*}
Find a similar transformation for $(F \vee G) \rightarrow H$. 

\begin{eqnarray*} 
(F \vee G) \rightarrow H 
                         \sim & \neg H \rightarrow (\neg F \wedge \neg G) \\
                         \sim & (\neg H \wedge \neg F) \rightarrow (\neg H \wedge \neg G) \\  
                         \sim & (H \vee G) \rightarrow (H \vee F)
\end{eqnarray*}


\item[\textbf{Problem 8}] We know that conjunction distributes over disjunction and that disjunction distributes over conjunction: 
\begin{gather*}
 F \wedge (G \vee H) \sim (F \wedge G) \vee (F \wedge H) \\
 F \vee (G \wedge H) \sim (F \vee G) \wedge (F \vee H)
\end{gather*}
Do these connectives distribute over equivalence? \\

For $\wedge$ No. \\
For $\vee$ Yes. $ F \vee (G \leftrightarrow H) \sim (F \vee G) \leftrightarrow (F \vee H)$ 

\item[\textbf{Problem 9}] De Morgan's laws
\begin{gather*}
 \neg ( F \wedge G) \sim \neg F \vee \neg G \\
 \neg ( F \vee G) \sim \neg F \wedge \neg G
\end{gather*}
show how to transform a formula of the form $\neg (F \odot G)$ when $\odot$ is $\wedge$ or $\vee$. Find similar transformations for the cases when $\odot$ is $\rightarrow$ or $\leftrightarrow$. (use truth table)
 \begin{gather*}
     \neg (F \rightarrow G) \sim F \wedge \neg G   \\
     \neg (F \leftrightarrow G) \sim \left\{
        \begin{array}{c}      
        F \leftrightarrow \neg G\\
        \neg F \leftrightarrow G
        \end{array}\right.
 \end{gather*}

\item[\textbf{Problem 10}] To simplify a formula means to find an equivalent formula that is shorter. Simplify the formulas: 
\begin{gather*} 
 F \vee ( F \wedge G) \sim F \\
 F \wedge (F \vee G) \sim F \\
 F \vee (\neg F \wedge G) \sim F \vee G
\end{gather*}
 \end{enumerate}
 
 \newpage
\section*{Adequate Sets of Connectives \colorbox{green}{$~$ }}
\begin{enumerate}
\item[\textbf{Problem 11}] For any formula, there exists an equivalent formula that contains no connectives other than (i) $\wedge$ and $\neg$; (ii) $\vee$ and $\neg$.  (structural induction)
    \begin{enumerate} [(i)]
        \item $\wedge$ and $\neg$ 
            \begin{itemize}
                \item There is no connective in atoms. Thus every atom has property $P$. 
                \item $\top \sim p \vee \neg p \sim \neg (p \wedge \neg p)$; $\bot \sim p \wedge \neg p$. Thus $\top$ and $\bot$ has property $P$. 
                \item By induction hypothesis, there exist $F' \sim F$, that contains no connective other than $\wedge$ \text{and} $\neg$.  Clearly, $\neg F \sim \neg F'$.  So if a formula $F$ has property P then so does $\neg F$. 
                \item $F \wedge G \sim F' \wedge G'$; \\
                 $F \vee G \sim F' \vee G' \sim \neg (\neg F' \wedge \neg G')$; (\textit{De Morgan's Law}) \\
                 $ F \rightarrow G \sim F' \rightarrow G' \sim \neg (F' \wedge \neg G')$;\\
                  Thus, for any binary connective $\odot$, if formulas $F$ and $G$ has property $P$, then so does $(F \odot G)$. 
            \end{itemize}
       Then we can conclude that property $P$ holds for all formulas.   
       
          \item $\vee$ and $\neg$ 
            \begin{itemize}
                \item There is no connective in atoms. Thus every atom has property $P$. 
                \item $\top \sim p \vee \neg p$; $\bot \sim \neg (p \vee \neg p) $. Thus $\top$ and $\bot$ has property $P$. 
                \item By induction hypothesis, there exist $F' \sim F$, that contains no connective other than $\wedge$ \text{and} $\neg$.  Clearly, $\neg F \sim \neg F'$.  So if a formula $F$ has property P then so does $\neg F$.
                \item $F \wedge G \sim F' \wedge G' \sim \neg (\neg F' \vee \neg G') $; (\textit{De Morgan's Law}) \\
                 $F \vee G \sim F' \vee G'$;  \\
                 $ F \rightarrow G \sim F' \rightarrow G' \sim \neg (F' \wedge \neg G') \sim \neg F' \vee G'$ \\
                 Thus, for any binary connective $\odot$, if formulas $F$ and $G$ has property $P$, then so does $(F \odot G)$. 
            \end{itemize}
       Then we can conclude that property $P$ holds for all formulas.    
    \end{enumerate}
    
\item[\textbf{Problem 12}] For any formula, there exists an equivalent formula that contains no connectives other than (i) $\rightarrow $ and $\neg$; (ii) $\rightarrow $ and $\bot$.  (structural induction)
    \begin{enumerate} [(i)]
        \item $\rightarrow $ and $\neg$ 
            \begin{itemize}
                \item There is no connective in atoms. Thus every atom has property $P$. 
                \item $\top \sim p \rightarrow p$; $\bot \sim \neg(p \rightarrow p)$. Thus $\top$ and $\bot$ has property $P$. 
                \item By induction hypothesis, there exist $F' \sim F$, that contains no connective other than $\rightarrow $ \text{and} $\neg$. Clearly, $\neg F \sim \neg F'$.  So if a formula $F$ has property P then so does $\neg F$. 
                \item $F \wedge G \sim F' \wedge G' \sim F' \wedge \neg (\neg G') \sim \neg ( F' \rightarrow \neg G') $; \\
                 $F \vee G \sim F' \vee G' \sim \neg (\neg F' \wedge \neg G') \sim \neg 
                 F' \rightarrow G' $ ;  \\
                 $F \rightarrow G \sim F' \rightarrow G'$;\\
                  Thus, for any binary connective $\odot$, if formulas $F$ and $G$ has property $P$, then so does $(F \odot G)$. 
            \end{itemize}
       Then we can conclude that property $P$ holds for all formulas.   
       
          \item $\rightarrow $ and $\bot$ 
            \begin{itemize}
                \item There is no connective in atoms. Thus every atom has property $P$. 
                \item $\top \sim p \rightarrow p \: (\bot \rightarrow p) $; $\bot$ is trivial. Thus $\top$ and $\bot$ has property $P$. 
                \item By induction hypothesis, there exist $F' \sim F$, that contains no connective other than $\rightarrow $ \text{and} $\bot$.  Clearly, $\neg F \sim \neg F' \sim F' \rightarrow \bot $.  So if a formula $F$ has property P then so does $\neg F$.
                \item $F \wedge G \sim F' \wedge G' \sim \neg ( F' \rightarrow \neg G') \sim [F' \rightarrow (G' \rightarrow \bot)] \rightarrow \bot ] $; \\
                 $F \vee G \sim F' \vee G' \sim \neg F' \rightarrow G' \sim (F' \rightarrow \bot) \rightarrow G'$;  \\
                 $ F \rightarrow G \sim F' \rightarrow G' $\\
                 Thus, for any binary connective $\odot$, if formulas $F$ and $G$ has property $P$, then so does $(F \odot G)$. 
            \end{itemize}
       Then we can conclude that property $P$ holds for all formulas.    
    \end{enumerate}

\item[\textbf{Problem 13}] Any propositional formula equivalent to $\neg p$ contains $\neg$ or $\bot$.   \begin{enumerate}[Prove 1:]
\item A formula without $\neg$ nor $\bot$ can be evaluated as $\top$ under certain interpretation. Thus, it can not be equivalent to $\neg p$. Thus, any propositional formula equivalent to $\neg p$ contains $\neg$ or $\bot$.
\item Suppose our alphabet is $p, \wedge, \vee, \rightarrow \top$. Let $B(F)$ means that F contains neither $\neg p$ nor $ \bot$.  \\
Clearly, we have $B(p), B(\bot)$. \\
Suppose we have $B(F)$ and $B(G)$, we can prove that $B( F \odot G)$. 
\end{enumerate}


\end{enumerate}


\newpage
\section*{Normal Forms \colorbox{yellow}{Note 3, page 5}}
A \textit{literal} is an atom or the negation of an atom. A propositional formula is said to be \textit{negation normal form} if 
\begin{itemize}
    \item it contains no connectives other than conjunction, disjunction, and negation, and 
    \item every negation in it is part of a literal. 
\end{itemize}

\begin{enumerate}
\item[\textbf{Problem 14}] Any formula is equivalent to a formula in negation normal form.  \\
Use structure induction. 
\begin{itemize}
\item $F$ is atom $a$, $\neg F = \neg a$, both $F$ and $\neg F$ are in NNF.  
\item if $F$ is $\bot$, $ F \sim p \wedge \neg p$, $\neg F \sim p \vee \neg p$. Similar for the case where $F = \top$. Both $F$ and $\neg F$ are in NNF. 
\item Let $F$ and $\neg F$ have NNF. The negation of them are still NNF. 
\item Suppose $F, \neg F, G, \neg G$ have NNF, this is to say, we have $F \sim F', \neg F \sim F'', G \sim G', \neg G \sim G''$, where $F', F'', G', G''$ are in NNF. 
    \begin{itemize}
        \item $F \wedge G \sim F' \wedge G'$,  \quad $\neg F \wedge \neg G \sim F'' \wedge G''$;
        \item $F \vee G \sim F' \vee G'$,  \quad $\neg F \vee \neg G \sim F'' \vee G''$;
        \item $F \rightarrow G \sim \neg F \vee G \sim F'' \vee G'$ ; \quad $\neg F \rightarrow \neg G \sim F \vee \neg G \sim F' \vee G''$. 
        
    \end{itemize}
\end{itemize}

A $simple~ conjunction$ is a formula of the form $L_1 \wedge \cdots \wedge L_n (n \geq 1)$, where $L_1, \ldots, L_n$ are literals. A formula is in \textit{disjunctive normal form (DNF)} if it has the form $C_1 \vee \cdots \vee C_m (m \geq 1)$, where $C_1, \ldots C_m$ are simple conjunctions. 


\item[\textbf{Problem 15}] Any formula is equivalent to a formula in disjunctive normal form.  
\begin{enumerate}[Prove 1:]
    \item Prove by using Problem 5.
    \item Prove like problem 14, use structure induction. NNF $\rightarrow$ DNF. 
\end{enumerate}

A $simple~ disjunction$ is a formula of the form $L_1 \vee \cdots \vee L_n (n \geq 1)$, where $L_1, \ldots, L_n$ are literals. Simple disjunctions are also called \textit{clauses}. 
A formula is in \textit{conjunctive normal form (CNF)} if it has the form $D_1 \wedge \cdots \wedge D_m (m \geq 1)$, where $D_1, \ldots D_m$ are simple disjunctions. 


\item[\textbf{Problem 16}] Let $F$ be a formula in disjunctive normal form. Show that $\neg F$ is equivalent to a formula in conjunctive normal form. \\
Since $F$ is in DNF, $F = C_1 \vee \cdots \vee C_m (m \geq 1)$, where $C_i = L_1 \wedge \cdots \wedge L_n$. \\
Thus, $\neg F \sim \neg C_1 \wedge \cdots \wedge \neg C_m (m \geq 1)$, where $\neg C_i \sim \overline{L_1} \vee \cdots \vee \overline{L_n}$. \\
Define $D_i = \neg C_i$, thus, $\neg F$ has the form $D_1 \wedge \cdots \wedge D_m (m \geq 1)$, where $D_1, \ldots, D_m$ are simple disjunctions. 

\item[\textbf{Problem 17}] Any formula is equivalent to a formula in conjunctive normal form.   \\
$\forall F $ \\
$~\quad \sim \qquad$  \{$\forall I, F^I = \neg (\neg (F^I)) = (\neg \neg F)^I$ \}\\
    $~\qquad \neg \neg F$  \\
$~\qquad \sim \qquad$   \{by Problem 15, $\neg F \sim G $ in DNF\}  \\ 
    $~\qquad \quad $ $\sim \neg G$  \\
 $~\qquad \quad \iff$   \{by Problem 16, $\neg G \sim H$ in CNF\}  \\ 
    $~\qquad \qquad $ $\sim H$ 
\end{enumerate}

\section*{Satisfiability and Entailment \colorbox{green}{}}
A \textit{set} $\Gamma$ is \textit{satisfiable} if there exists an interpretation that satisfies all formulas in $\Gamma$, and \textit{unsatisfiable} otherwise. 

\begin{enumerate}
\item[\textbf{Problem 18}] A set $\Gamma$ of literals is satisfiable iff there is no atom $A$ for which both $A$ and $\neg A$ belong to $\Gamma$. 

Left to Right: If a set $\Gamma$ of literals is satisfiable, then there exists an interpretation $I$ that satisfies all formulas in $\Gamma$. Clearly, for any atom $A$, $A^I$ and $(\neg A)^I$ can not be satisfied at the same time. Thus, there is no atom $A$ for which both $A$ and $\neg A$ belong to $\Gamma$. \\
Right to Left: If there is no atom $A$ for which both $A$ and $\neg A$ belong to $\Gamma$. Thus, we can find interpretation $I$ for any atom $A$ based on the following rule:
\begin{gather*}
A^I =  \left\{
        \begin{array}{c}      
        t \qquad \text{if} \: A \in \Gamma \\
        f \qquad \text{if} \: \neg A \in \Gamma \\
        f \qquad \text{otherwise}
        \end{array}\right.
\end{gather*}
Thus for any literal $L \in \Gamma$, we have $I(L) = t$. Clearly, this set $\Gamma$ of literal is satisfiable. 

\end{enumerate}







\noindent A set $\Gamma$ of formulas \textit{entails} a formula $F$ (symbolically, $\Gamma \models F$), if every interpretation that satisfies all formulas in $\Gamma$ satisfies $F$ also. 

\begin{enumerate}
\item[\textbf{Problem 19}] For any formulas $F_1, \ldots, F_n, G$, the following conditions are equivalent:
\begin{enumerate}[(1)]
\item $F_1, \ldots, F_n \models G$, 
\item $(F_1 \wedge \cdots \wedge F_n) \rightarrow G$ is tautology, 
\item the set $\{F_1, \ldots, F_n, \neg G \}$ is unsatisfiable. 
\end{enumerate}
\begin{enumerate}[(1)]
\item By definition, $\forall I s.t. F_1^I = t, \ldots, F_n^I = t$, we have $G^I = t$. 
\item $(F_1 \wedge \cdots \wedge F_n) \rightarrow G$ is tautology mean that $\forall I s.t. (F_1 \wedge \cdots \wedge F_n)^I = t, G^I = t$. From problem 3, we know that $(F_1 \wedge \cdots \wedge F_n)^I = t \sim F_1^I = t, \ldots, F_n^I = t$. This turn out to be (1). Thus we have $(1) \Longleftrightarrow (2)$. 
\item the set $\{F_1, \ldots, F_n, \neg G \}$ is unsatisfiable means that there $\not \exists I s.t. F_1^I = t, \ldots, F_n^I = t, \neg G^I = t$. This is to say, $\forall I s.t. F_1^I = t, \ldots, F_n^I = t$, we have $\neg G^I = f$. Thus, $G^I = t$. This turn out to be (1). Thus we have $(1) \Longleftrightarrow (3)$. 

\end{enumerate}





\end{enumerate}

\section*{Clausification \colorbox{yellow}{}}
To \textit{clausify} a formula $F$ means to find a formula 
$F'$ that may contain some new atoms, not occurring in $F$, such that
\begin{itemize}
\item $F'$ is in conjunctive normal form. 
\item any interpretation satisfying $F'$ satisfies $F$, and 
\item any interpretation satisfying $F'$ can be extended to the new atoms so that it will satisfy $F'$. 
\end{itemize}
Here is an algorithm for clausifying a propositional formula: \\
\begin{algorithm}[H]
 \Begin{
     $\Gamma \leftarrow \emptyset$ \\
     \While{F is not CNF}{
          $A \leftarrow$ a new atom \\
          $G \leftarrow$ a minimal non-literal subformula of F \\
          $F \leftarrow$ the result of replacing G in F by A \\
          $\Delta \leftarrow$ the set of clauses of the CNF of $A
               \leftrightarrow  G$ \\
          $\Gamma \leftarrow \Gamma \cup \Delta$
      }
      \Return the conjunction of F with clauses $\Gamma$
}
\end{algorithm}

\begin{enumerate}
\item[\textbf{Problem 20}] (i) Apply this algorithm to the formula $p \vee \neg (q \rightarrow r)$. (ii) Determine whether this formula is equivalent to the result of its clausification. \\
The formula is not equivalent to the result of its clausification for the latter has new atom whose interpretation is not defined in the interpretation for the original formula. 
\end{enumerate}
\newpage

\section*{Positive Programs \colorbox{yellow}{}}
A \textit{positive rule} is a propositional formula of the form $F \rightarrow G$, where $F$ and $G$ contain no connectives other than $\top, \bot, \wedge, \text{and} \vee$. The antecedent $F$ is called the \textit{body} of the rule, and the consequent $G$ is called its \textit{head}. Rules are often written with the head on the left and body on the right: $G \leftarrow G$. A rule of the form $G \leftarrow \top$ is often identified with its head $G$. Rules of the form $\bot \leftarrow F$ are called \textit{constraints} and are usually written as $\leftarrow F$. 

A \textit{positive (logic) program} is a set of positive rules. 

A positive rule is \textit{flat} if (i) its head is $\bot$, or an atom, or a disjunction of several atoms, and (ii) its body is $\top$, or an atom, or a conjunction of several atoms. Any propositional formula can be transformed into an equivalent set of flat positive rules by converting it to conjunctive normal form and then rewriting each of its simple disjunctions
\begin{equation*}
A_1 \vee \cdots A_m \vee \neg A_{m + 1} \vee \cdots \vee A_n
\end{equation*}
as the rule
\begin{equation*}
A_1 \vee \cdots A_m \vee \leftarrow A_{m + 1} \wedge \cdots \wedge A_n
\end{equation*}
*The expression in the head is understood as $\bot$ if $m = 0$; the expression in the body is understood as $\top$ if $m = n$. 

\section*{Minimal Models \colorbox{yellow}{}}
In the theory of logic programs it is customary to identify an interpretation with the set of atoms to which it assigns the value $t$. For instance, the interpretation that assigns $t$ to the atom $p$ and $f$ to all other atoms can be viewed as the singleton {p}. 

A \textit{model} of a set $\Gamma$ of formulas is an interpretation that satisfies all formulas in $\Gamma$. A model $M$ of $\Gamma$ is \textit{minimal} if no proper subset of $M$ is a model of $\Gamma$. For example, the program
\begin{equation}
\begin{gathered}
p, \\
q \leftarrow r
\end{gathered}
\end{equation}
has 3 models:
\begin{equation*}
 \{p\}, \{p, q\}, \{p, p, r\};
\end{equation*}
only the first of them is minimal. 

\begin{enumerate}
\item[\textbf{Problem 21}] Find all models of the program
\begin{gather*}
p\vee q, \\
q \vee r, \\
q \leftarrow p \wedge r. 
\end{gather*}
Determine which of these models are minimal. \\
\begin{gather*}
 \{p, q, r\} \\
 \{p,q\}, \{q, r\} \\
  \{q\}
\end{gather*}




\item[\textbf{Problem 22}] Consider the positive program consisting of the rules $A \vee B$ for all pairs of distinct atoms $A, B$. If the total number of atoms is $n$, then how many models does this program have? How many of them are minimal? \\
Claim: A model of the above positive program must have at least $n - 1$ atoms. 
Suppose we have a model $M$ who has less than $n - 1$ atoms. $\exists$ two atoms, $A$ and $B$, where $A \vee B$ is one of the rules while $A \not \in M, B \not \in M$. Thus, $(A \vee B)^M = A^M \vee B^M = f \vee f = f$. Thus, $M$ failed to satisfy all rules. Contradiction. \\
From the above claim, we can see that the minimal model should have $n - 1$ atoms. $C_{n}^{n - 1} = n$. There is one more model which contains all the atoms. Thus, the total number of models is $n$. 


\item[\textbf{Problem 22}] Let $\Gamma$ be a positive program such that the head of each of its rules is an atom. Show that the intersection of all models of $\Gamma$ is the only minimal model of $\Gamma$. \\
Claim: Given a positive program $\Gamma$ such that the head of each of its rules is an atom, for two model $M_1, M_2$, $M_1 \cap M_2$ is model as well.  \\
Prove: For the head $p$ of each rule in the form $p \leftarrow q_1 \wedge \cdots \wedge q_n$ :
\begin{itemize}
\item If $p \in M_1 \cap M_2$, then the rule is satisfied. 
\item Otherwise, suppose $p \not \in M_i$, then $\exists j, q_j \not \in M_i, q_j \not \in M_1 \cap M_2$. Thus, the rule is also satisfied.
\end{itemize}
For the head $p$ of each rule in the form $p \leftarrow \top$: 
$p \in M_1, p \in M_2$, thus $p \in M_1 \cap M_2$, the rule is satisfied under $M_1 \cap M_2$\\
By induction, $\cap m_i$ is model, which is also the subset of all models. Thus, the intersection of all models of $\Gamma$ is the only minimal model. 
\end{enumerate}



\noindent If $\Gamma_1, \Gamma_2$ are sets of formulas such that $\Gamma_1 \subseteq \Gamma_2$ then every model of $\Gamma_2$ s a model of $\Gamma_1$. But we cannot assert, in general, that every minimal model of $\Gamma_2$ is a minimal model of $\Gamma_1$. For instance, if we add the rule $r \leftarrow p$ to program (1) then it will get a new minimal model, $\{p, q, r\}$. In this sense, the concept of a minimal model is "nonmonotonic."

\begin{enumerate}
\item[\textbf{Problem 24}] Let $F$ be a formula containing no connectives other than $\top, \bot, \wedge, \vee$, and let $M_1, M_2$ be sets of atoms such that $M_1 \subseteq M_2$. Show that if $M_1$ satisfies $F$ then so does $M_2$. 

Idea: Structural Induction \\

\begin{itemize}
\item If $F$ is an atom, $M_1$ satisfies $F$ then so does $M_2$;
\item $\top, \bot$ is trivial;
\item There is no $\neg$ in F, we don't need to consider this case;
\item Suppose given $M_1$ satisfy both $F$ and $G$, and $M_1 \subseteq M_2$, we have $M_2$ satisfy both $F$ and $G$ as well. 
    \begin{itemize}
        \item For $F \wedge G$, $(F \wedge G)^{M_2} = \wedge (F^{M_2}, G^{M_2}) = t$, thus, induction hypothesis holds true;
        \item For $F \vee G$, $(F \vee G)^{M_2} = \vee (F^{M_2}, G^{M_2}) = t$, thus, induction hypothesis holds true;
        \item there is no $\rightarrow$ in F, we don't need to consider this case. 
    \end{itemize}
\end{itemize}
Thus, we are done. 


\item[\textbf{Problem 25}] For any positive program $\Gamma$ and any constraint $\leftarrow F$, an interpretation $M$ is a \textbf{minimal} model of $\Gamma \cup \{\leftarrow F \}$ iff $M$ is a \textbf{minimal} model of $\Gamma$ and does not satisfy $F$. 
\begin{itemize}
    \item[$\leftarrow$] If $M$ is a \textbf{minimal} model of $\Gamma$ and does not satisfy $F$, prove that $M$ is a \textbf{minimal} model of $\Gamma \cup \{\leftarrow F \}$.
     \begin{itemize}
         \item $M$ is a model of $\Gamma \cup \{\leftarrow F \}$. Clearly, $M$ satisfies $\Gamma$. $M$ doesn't satisfy $F$, thus $M$ satisfy $\{\leftarrow F \}$. Thus,  $M$ satisfies $\Gamma \cup \{\leftarrow F \}$.
         \item $M$ is a minimal model of $\Gamma \cup \{\leftarrow F \}$. Suppose there is a subset $M'$ of $M$ which is a model for $\Gamma \cup \{\leftarrow F \}$ as well. Thus, $M'$ must also be a model of $\Gamma$. However, we already know that $M$ is a minimal model of $\Gamma$. Contradiction. Thus, $M$ is a minimal model of $\Gamma \cup \{\leftarrow F \}$. 
     \end{itemize}
     \item[$\rightarrow$] If $M$ is a \textbf{minimal} model of $\Gamma \cup \{\leftarrow F \}$, prove that $M$ is a \textbf{minimal} model of $\Gamma$ and does not satisfy $F$. 
     \begin{itemize}
         \item Sicne $\{\leftarrow F \}^M = t$, $F^M = f$, thus, $M$ does not satisfy $F$.
         \item Clearly, $M$ must be a model of $\Gamma$. 
         \item Suppose there is a subset $M'$ of $M$ which is a model for $\Gamma$. From the counter reverse of Problem 24, $M'$ doesn't satisfy $F$ as well. Thus, from $leftarrow$, we know that $M'$ is a model of $\Gamma \cup \{\leftarrow F \}$. However, we already know that $M$ is a \textbf{minimal} model of $\Gamma \cup \{\leftarrow F \}$. Contradiction. Thus, $M$ is a \textbf{minimal} model of $\Gamma$. 
     \end{itemize}
\end{itemize}



\item[\textbf{Problem 26}] If $M$ is a minimal model of a positive program $\Gamma$ then every atom from $M$ occurs in the head of one of the rules of $\Gamma$. 

Idea: Suppose $M$ is a minimal model and one of the atoms from $M$ does not occur in the head of any of the \textbf{flat} rules. By removing such atom from M, we will not change the interpretation of each head. However, the body can't be changed from false to true (The body can only be $\top$, or an atom, or a conjunction of several atoms). For all other cases, the new model satisfies $\Gamma$, thus, $M$ is not a minimal model. Contradiction. 

Prove by contradiction: Suppose there is an atom $p$ in minimal model $M$ that doesn't occur in the head of any of the rules. Define $M' = M \backslash \{p\}$. \\
For any rule $H \leftarrow B$ in $\Gamma$, if 
    \begin{itemize}
        \item if $M$ satisfies $H$, $M'$ satisfies $H$ as well, so $M'$ satisfies $H \leftarrow B$. 
        \item if $M$ doesn't satisfy $H$, $M'$ doesn't neither. Since $M$ satisfies $H \leftarrow B$, $M$ must don't satisfy $B$. From Problem 24, $M'$ doesn't satisfy $B$ as well.  Thus, $M'$ satisfy rule $H \leftarrow B$. Since we already know $M$ is a minimal model. Contradiction. 
    \end{itemize}



\end{enumerate}


\end{document}























