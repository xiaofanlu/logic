\documentclass[12pt]{article}
\pagestyle{plain}

\usepackage{graphicx}
\usepackage{float}
\usepackage{times}
\usepackage{color}
\usepackage{listings}
\usepackage{enumerate}
\usepackage{amsmath}
\usepackage{amsthm}
\usepackage{fancyhdr}
\usepackage{titlesec}
\usepackage{ascii}
\usepackage[T1]{fontenc}
\usepackage{datetime}
\usepackage[top=1.5in, bottom=1.5in, left=1in, right=1in]{geometry}

\usepackage{algorithm2e}
  
\providecommand{\e}[1]{\ensuremath{\times 10^{#1}}}

\pagestyle{fancy} 
\renewcommand{\headrulewidth}{0pt}
\shortdate
\fancyhead[LE, RO]{\textit{\date{\currenttime} \today}}
\fancyhead[LO]{\textit {Xiaofan Lu}}

\newcommand{\xfc}[1]{
\stepcounter{#1}
\arabic{#1}
}

\newcommand{\xfp}[1]{
\addtocounter{#1}{-1}
\arabic{#1}
\addtocounter{#1}{1}
}

\newcommand{\xfl}[2]{
\xfc{c}.  & $#1$                               & $\qquad$ & -- #2. \\
}



\titleformat{\section}{\bf\Large\centering}{\thesection}{1em}{}
  
\begin{document}
\title{CS388L: Introduction to Mathematical Logic \\ Quiz 6, Due April 15}
\author{Xiaofan Lu, xl4326}
\date{\vspace{-3ex}}
\maketitle

\noindent
The following rule 
\begin{equation*}
\frac{\Gamma \Rightarrow F \rightarrow G \qquad \Delta \Rightarrow \neg G}{\Gamma, \Delta \Rightarrow \neg F}
\end{equation*}
is sound in $G_3$. \\\\
\noindent
We need to show that if 
\begin{eqnarray}
\label{delta}
\Delta \rightarrow \neg G \\
\label{gamma}
\Gamma \rightarrow (F \rightarrow G)
\end{eqnarray}
are tautological in $G_3$, then
\begin{equation}
\Gamma \wedge \Delta \rightarrow \neg F
\end{equation}
is also tautological. 

\begin{proof}
\eqref{delta} is tautological means that $\forall I$, we have 
\begin{equation*}
(\Delta \rightarrow \neg G)^I = \rightarrow (\Delta^I, \neg G^I) = 1
\end{equation*}
From definition, we know that 
\begin{equation}
\label{rule1}
\Delta^I \leq \neg G^I
\end{equation}

\eqref{gamma} is tautological means that $\forall I$, we have 
\begin{equation*}
(\Gamma \rightarrow (F \rightarrow G))^I = \rightarrow (\Gamma^I, (F \rightarrow G)^I) = 1
\end{equation*}
From definition, we know that 
\begin{equation}
\label{rule2}
\Gamma^I \leq (F \rightarrow G)^I
\end{equation}
Consider the following two cases:
\begin{itemize}
\item if $F^I \leq G^I$, then $(F \rightarrow G)^I = \rightarrow (F^I, G^I) = 1$. Clearly, \eqref{rule2} holds true. We have:
\begin{equation}
\label{case_1}
\neg F^I \geq \neg G^I
\end{equation}
This is because if both $F^I$ and $G^I$ are greater than 0, then $\neg F^I = \neg G^I = 0$. If both $F^I$ and $G^I$ equal 0, then $\neg F^I = \neg G^I = 1$. If only one of them equals 0, it must be $0 = F^I < G^I$, then $0 = \neg G^I < \neg F^I = 1$. \\
Combining \eqref{rule1} and \eqref{case_1}, we have:
\begin{equation*}
\Delta^I \leq \neg G^I \leq \neg F^I
\end{equation*}
Thus, 
\begin{equation*}
\min(\Gamma^I, \Delta^I) \leq \neg F^I
\end{equation*}


\item if $F^I > G^I$, then $(F \rightarrow G)^I = \rightarrow (F^I, G^I) = G^I$. From \eqref{rule1} and  \eqref{rule2}:
\begin{gather*}
 \left\{
        \begin{array}{c}      
        \Delta^I \leq \neg G^I \\
        \Gamma^I \leq \:\:\:G^I
        \end{array}\right.
\end{gather*}
Clearly, one of $G^I$ and $\neg G^I$ must be zero. (If $G^I = 0$, the claim holds. If $G^I > 0$, then $\neg G^I = 0$.) \\
Thus, 
\begin{equation*}
\min(\Gamma^I, \Delta^I) = 0 \leq \neg F^I 
\end{equation*}
\end{itemize}
From both cases, we have $\min(\Gamma^I, \Delta^I) \leq \neg F^I$, this is the same as:
\begin{eqnarray*}
(\Gamma \wedge \Delta)^I \leq \neg F^I \\
\rightarrow ((\Gamma \wedge \Delta)^I, \neg F^I) = 1\\
(\Gamma \wedge \Delta \rightarrow \neg F)^I = 1
\end{eqnarray*}
Thus, $\Gamma \wedge \Delta \rightarrow \neg F$ is tautological in $G_3$. 
\end{proof}


\end{document}























