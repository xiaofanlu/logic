\documentclass[12pt]{article}
\pagestyle{plain}

\usepackage{graphicx}
\usepackage{float}
\usepackage{times}
\usepackage{color}
\usepackage{listings}
\usepackage{enumerate}
\usepackage{amsmath}
\usepackage{amsthm}
\usepackage{fancyhdr}
\usepackage{titlesec}
\usepackage{ascii}
\usepackage[T1]{fontenc}
\usepackage{datetime}
\usepackage[top=1.5in, bottom=1.5in, left=1in, right=1in]{geometry}

\usepackage{algorithm2e}
  
\providecommand{\e}[1]{\ensuremath{\times 10^{#1}}}

\pagestyle{fancy} 
\renewcommand{\headrulewidth}{0pt}
\shortdate
\fancyhead[LE, RO]{\textit{\date{\currenttime} \today}}
\fancyhead[LO]{\textit {Xiaofan Lu}}

\newcommand{\xfc}[1]{
\stepcounter{#1}
\arabic{#1}
}

\newcommand{\xfp}[1]{
\addtocounter{#1}{-1}
\arabic{#1}
\addtocounter{#1}{1}
}

\newcommand{\xfl}[2]{
\xfc{c}.  & $#1$                               & $\qquad$ & -- #2. \\
}



\titleformat{\section}{\bf\Large\centering}{\thesection}{1em}{}
  
\begin{document}
\title{CS388L: Introduction to Mathematical Logic \\ Quiz 5, Due April 1}
\author{Xiaofan Lu, xl4326}
\date{\vspace{-3ex}}
\maketitle

\noindent
Prove by natural deduction:
\begin{equation*}
    ((p \rightarrow q) \rightarrow p) \rightarrow p
\end{equation*}

\begin{proof} $~$\\
\newcounter{c}
\begin{table}[H]
\begin{center}
\begin{tabular}{llll}
A1. & $(p \rightarrow q) \rightarrow p$.                    & $\qquad$ & \\
\xfl{A1 \Rightarrow (p \rightarrow q) \rightarrow p}        {axiom}
\xfl{\Rightarrow p \vee \neg p}                             {axiom}
A2. & $p$.                                                  & $\qquad$ & \\
\xfl{A2 \Rightarrow p}                                      {axiom}
%\xfl{A2, p \Rightarrow q }                                  {$(W), \xfp{c}$}
%\xfl{A2 \Rightarrow p \rightarrow q }                       {$(\rightarrow I), \xfp{c}$}
%\xfl{A1, A2 \Rightarrow p}                                  {$(\rightarrow E), \xfp{c}, 1$}
A3. & $\neg p$.                                             & $\qquad$ & \\
\xfl{A3 \Rightarrow \neg p}                                 {axiom}
\xfl{A3, A2 \Rightarrow \bot}                                {$(\neg E), 3, \xfp{c}$}
\xfl{A3, A2 \Rightarrow q}                                   {$(C), \xfp{c}$}
\xfl{A3 \Rightarrow p \rightarrow q}                        {$(\rightarrow I), \xfp{c}$}
\xfl{A1, A3 \Rightarrow p}                                  {$(\rightarrow E), \xfp{c}, 1$}
\xfl{A1 \Rightarrow p}                                     {$(\vee E), 2, 3, \xfp{c}$}
\xfl{\Rightarrow  ((p \rightarrow q) \rightarrow p) \rightarrow p}{$(\rightarrow I), \xfp{c}$}
\end{tabular}
\end{center}
\end{table}
\end{proof}

\end{document}























